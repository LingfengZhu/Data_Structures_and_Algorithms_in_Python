\section{线性方程组的敏度分析}

%%%%
%%%%
\begin{frame}\ft{\secname}

\textcolor{acolor5}{线性方程组的敏感性问题}

考察线性方程组
$$
Ax=b,
$$
若给$A$和$b$以微小的扰动,其解会有何影响。

\end{frame}

%%%%
%%%%
\begin{frame}\ft{\secname}

\begin{li}
$$
\left[
\begin{array}{cc}
2.0002 & 1.9998\\
1.9998 & 2.0002
\end{array}
\right]
\left[
\begin{array}{c}
x_1\\
x_2
\end{array}
\right]
=
\left[
\begin{array}{c}
4\\
4
\end{array}
\right]
\pause
\textcolor{acolor3}{~\rightarrow ~x=(1~1)^T}.
$$
\pause
$$
\left[
\begin{array}{cc}
2.0002 & 1.9998\\
1.9998 & 2.0002
\end{array}
\right]
\left[
\begin{array}{c}
x_1\\
x_2
\end{array}
\right]
=
\left[
\begin{array}{c}
4{+0.0002}\\
4{-0.0002}
\end{array}
\right]~~
\pause
\textcolor{acolor3}{~\rightarrow ~\tilde{x}=(1.5~0.5)^T}.
$$
\end{li}

\pause 

$$
\frac{\|\tilde{x}-x\|_{\infty}}{\|x\|_{\infty}}=\frac12,~~~~~
\frac{\|\delta b\|_{\infty}}{\|b\|_{\infty}}=\frac1{20000}.
$$



\end{frame}

%%%%
%%%%
\begin{frame}\ft{\secname}

考察非奇异线性方程组$Ax = b$。
$$
\begin{array}{l}
\left.
\begin{array}{l}
A \rightarrow A+\delta A\\[0.2cm]
b \rightarrow b+\delta b
\end{array}
\right\}
~\Longrightarrow~
x \rightarrow x + \delta x.
\end{array}
$$
\pause 
$$
(A+\delta A)(x + \delta x) = b + \delta b
$$
$$
\Longrightarrow (A+\delta A) \delta x  =  \delta b - \delta A x
$$
\pause 
\textcolor{acolor3}{
只要$\|A^{-1}\|\cdot\|\delta A\|<1$,就有$A+\delta A$可逆,并且
$$
\|(I + A^{-1}\delta A)^{-1}\|\le \frac1{1-\|A^{-1}\|\cdot \|\delta A\|}.
$$
}
\pause 
于是
$$
\delta x = (A+\delta A)^{-1} (\delta b - \delta A x)
= (I+A^{-1}\delta A)^{-1} A^{-1} (\delta b - \delta A x).
$$



\end{frame}

%%%%
%%%%
\begin{frame}\ft{\secname}

由
$$
\delta x 
= (I+A^{-1}\delta A)^{-1} A^{-1} (\delta b - \delta A x) 
$$
可得
$$
\begin{array}{rcl}
\|\delta x\| 
&\le&  
\| (I+A^{-1}\delta A)^{-1}\| \cdot \| A^{-1} \| (\|\delta b \| + \|\delta A\|\cdot \| x\|)\\[0.4cm]  
& \pause 
\le & \pause \ds \frac{\| A^{-1} \|}{1-\|A^{-1}\|\cdot \|\delta A\|}(\|\delta b \| + \|\delta A\|\cdot \| x\|)
\end{array}
$$
\pause 
由$ \|b\|\le \|A\|\cdot\|x\|$可知 \pause 
$$
\frac{\|\delta x\|}{\|x\|} \le \frac{\| A^{-1} \|\cdot \| A \|}{1-\|A^{-1}\|\cdot \|\delta A\|}
\left(\frac{\|\delta A \|}{\|A\|} + \frac{\|\delta b \|}{\|b\|}\right).
$$



\end{frame}

%%%%
%%%%
\begin{frame}\ft{\secname}

\begin{dingli}
\textcolor{acolor5}{条件:}
\begin{itemize}
\item 
$\|\cdot\|$满足条件$\|I\|=1$, $A\in \R^{n\times n}$非奇异,$b\in \R^{n}$非零
\item
$\delta A\in \R^{n\times n}$满足$\|A^{-1}\|\cdot\|\delta A\|<1$
\item
$$ \left\{
\begin{array}{cl}
Ax = b & {\rightarrow \mbox{解为}x} \\[0.2cm]
(A+\delta A)(x + \delta x) = b + \delta b & {\rightarrow \mbox{解为} x+\delta x}
\end{array}
\right.
$$
\end{itemize}
\pause 
\textcolor{acolor5}{结论:}
$$
\frac{\|\delta x\|}{\|x\|}
\le
\frac{\kappa(A)}{1-\kappa(A)\frac{\|\delta A\|}{\|A\|}}
\left(
\frac{\|\delta A \|}{\|A\|} + \frac{\|\delta b \|}{\|b\|}
\right),
$$


\end{dingli}





\end{frame}

%%%%
%%%%
\begin{frame}\ft{\secname}

$$
\begin{array}{l}
\frac{\|\delta A\|}{\|A\|}\mbox{较小} \\[0.3cm] \pause 
\Longrightarrow ~~ \frac{\kappa(A)}{1-\kappa(A)\frac{\|\delta A\|}{\|A\|}} \approx
\kappa(A) \\[0.3cm] \pause 
\Longrightarrow ~~
\textcolor{acolor3}{
\frac{\|\delta x\|}{\|x\|}
\le \kappa(A)\left(
\frac{\|\delta A \|}{\|A\|} + \frac{\|\delta b \|}{\|b\|}
\right)}
\end{array}
$$

\vspace{0.2cm}
\pause

\textcolor{acolor5}{结论}
\begin{itemize}
\item 
$x$的相对误差由$b$和$A$的相对误差之和放大$\kappa(A)$倍得来 \pause 
\item
若$\kappa(A)$不大,则扰动对解的影响也不会太大 \pause 
\item
若$\kappa(A)$很大,则扰动对解的影响可能会太大
\end{itemize}

\end{frame}

%%%%
%%%%
\begin{frame}\ft{\secname}

\begin{dingyi}[条件数]
$\kappa(A) =\| A^{-1} \|\cdot \| A \|$称为线性方程组$Ax=b$的\textcolor{acolor3}{条件数}。
\end{dingyi}
\vspace{0.2cm}
\pause

条件数在一定程度上刻画了扰动对解的影响程度。\pause 
\begin{itemize}
\item
{若$\kappa(A)$很大,则我们就说该线性方程组的求解问题是病态的,或者说$A$是病态的;} \pause 
\item
{若$\kappa(A)$很小,则我们就说该线性方程组的求解问题是良态的,或者说$A$是良态的。}
\end{itemize}


\end{frame}

%%%%
%%%%
\begin{frame}\ft{\secname}

\textcolor{acolor5}{条件数与范数有关},$\R^{n\times n}$上任意两种范数下的条件数$\kappa_{\alpha}(A)$与$\kappa_{\beta}(A)$都是等价的,即存在常数$c_1$和$c_2$,
使得
$$
c_1 \kappa_{\alpha}(A) \le \kappa_{\beta}(A) \le c_2 \kappa_{\alpha}(A).
$$

例如,
$$
\begin{array}{rcl}
\frac1n \kappa_2(A) \le & \kappa_1(A) & \le n \kappa_2(A), \\[0.2cm]
\frac1n \kappa_{\infty}(A) \le & \kappa_2(A) & \le n \kappa_{\infty}(A), \\[0.2cm]
\frac1{n^2} \kappa_1(A) \le & \kappa_{\infty}(A) & \le n^2 \kappa_2(A).
\end{array}
$$


\end{frame}

%%%%
%%%%
\begin{frame}\ft{\secname}

\begin{tuilun}
\textcolor{acolor5}{条件:}
\begin{itemize}
\item 
$\|\cdot\|$满足条件$\|I\|=1$,$A\in \R^{n\times n}$非奇异,$b\in \R^{n}$非零
\item
$\delta A\in \R^{n\times n}$满足$\|A^{-1}\|\cdot\|\delta A\|<1$
\end{itemize}
\pause 
\textcolor{acolor5}{结论}
\begin{itemize}
\item $A+\delta A$也非奇异
\item
$$
\frac{\|(A+\delta A)^{-1} - A^{-1}\|}{\|A^{-1}\|}
\le
\frac{\kappa(A)}{1-\kappa(A)\frac{\|\delta A\|}{\|A\|}}
\frac{\|\delta A \|}{\|A\|}.
$$
\end{itemize}

\end{tuilun}
\vspace{0.2cm}
\pause

\textcolor{acolor3}{这表明$\kappa(A) =\| A^{-1} \|\cdot \| A \| $可当做矩阵求逆问题的条件数。}


\end{frame}

%%%%
%%%%
\begin{frame}\ft{\secname}

\begin{tuilun}[条件数的几何意义]
设$A\in\R^{n\times n}$非奇异,则
$$
\min\left\{\frac{\|\delta A\|_2}{\|A\|_2}: A+\delta A\mbox{奇异}\right\}
= \frac1{\| A^{-1} \|_2\cdot \| A \|_2}
= \frac1{\kappa_2(A)},
$$
即在谱范数下,一个矩阵的条件数的倒数正好等于该矩阵与全体奇异矩阵所成集合的相对距离。
\end{tuilun}
\vspace{0.2cm}
\pause

\textcolor{acolor3}{当$A\in\R^{n\times n}$十分病态时,它与一个奇异矩阵非常接近。}


\end{frame}

%%%%
%%%%
\begin{frame}\ft{\secname}
\begin{zhengming}
只需证
$$
\min\left\{\|\delta A\|_2: A+\delta A\mbox{奇异}\right\}
= \frac1{\| A^{-1} \|_2}.
$$
\pause 
事实上,{当$\|A^{-1}\|_2\cdot\|\delta A\|_2<1$时,$A+\delta A$可逆},于是若$A+\delta A$奇异,则必有
$$
\|A^{-1}\|_2\cdot\|\delta A\|_2\ge 1
$$
\pause 
即
$$\textcolor{acolor5}{
\min\left\{\|\delta A\|_2: A+\delta A\mbox{奇异}\right\} \ge \frac1{\|A^{-1}\|_2}.
}
$$
\end{zhengming}

\end{frame}


%%%%
%%%%
\begin{frame}\ft{\secname}

\begin{zhengming}
存在$x\in\R^{n}$满足$\|x\|_2=1$使得$\|A^{-1}x\|_2=\|A^{-1}\|_2$。
令
$$
y = \frac{A^{-1}x}{\|A^{-1}x\|_2}, ~~
\delta A = - \frac{xy^T}{\|A^{-1}\|_2},
$$ \pause 
则有$\|y\|_2 =1$,且 \pause 
$$
\begin{array}{cc}
(A+\delta A)y = Ay + \delta Ay 
= \frac{x}{\|A^{-1}x\|_2} - \frac{x}{\|A^{-1}\|_2} = 0 \\[0.4cm] \pause 
\|\delta A\|_2 = \max_{\|z\|_2=1}\left\|\frac{xy^T}{\|A^{-1}\|_2} z\right\|_2
= \frac{\|x\|_2}{\|A^{-1}\|_2}\max_{\|z\|_2=1}|y^Tz|=\frac1{\|A^{-1}\|_2},  
\end{array}
$$
\pause 
这说明找到了一个$\delta A\in\R^{n\times n}$使得$A+\delta A$奇异,
且$\|\delta A\|_2 = \|A^{-1}\|_2^{-1}$。
\end{zhengming}
\end{frame}

