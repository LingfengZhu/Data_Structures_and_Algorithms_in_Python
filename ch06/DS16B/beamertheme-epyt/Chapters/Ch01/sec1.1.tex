\section{数值计算简介}

%%%%%%%%%%%%%%%%%
\begin{frame}\frametitle{\secname} %\fst{现代科学的三大手段}
当今世界科学活动的三种主要方式\\

\begin{itemize}
\item
理论
\item
实验
\item
科学计算
\end{itemize}
\end{frame}
%
%%%%%%%%%%%%%%%%%%
\tikzstyle{decision} = [diamond, draw, fill=blue!20,
text width=4.5em, text badly centered, node distance=3cm, inner sep=0pt]
\tikzstyle{block} = [rectangle, draw, fill=blue!20,
text width=2em, text centered, rounded corners, minimum height=4em]
\tikzstyle{line} = [draw, -latex']
\begin{frame}\ft{\secname}%\fst{解决科学工程问题的步骤}
\begin{figure}
\centering
\begin{tikzpicture}[scale=0.5, node distance = 2.0cm, auto]
  \pause
  \node [block] (problem) {实\\ 际\\ 问\\ 题};

  \pause

  \node [block, right of=problem] (model) {数\\ 学\\ 模\\ 型};
  \path [line] (problem) -- (model);
  \pause

  \node [block, right of=model] (method) {计\\算\\方\\法};
  \path [line] (model) -- (method);
  \pause

  \node [block, right of=method] (prog) {程\\ 序\\ 设\\ 计};
  \path [line] (method) -- (prog);
  \pause

  \node [block, right of=prog] (comp) {上\\ 机\\ 求\\ 解};
  \path [line] (prog) -- (comp);
  \pause

  \draw[decorate, decoration={brace, mirror}, very thick, blue] (0,-3.5) -- (4,-3.5);
  \draw (2,-4.5) node {应用数学};
  \pause

  \draw[decorate, decoration={brace, mirror}, very thick, blue] (8,-3.5) -- (16,-3.5);
  \draw (12,-4.5) node {计算数学};
\end{tikzpicture}
\caption{解决科学工程问题的步骤}
\end{figure}
\end{frame}
%
%%%%%%%%%%%%%%%%%
\begin{frame}\ft{\secname}\fst{研究数值方法的必要性}

\begin{li}
  求解线性方程组
  $$
  Ax = b, \quad A \in \R^{n\times n}, \quad b \in \R^n
  $$
\end{li}
\pause
\begin{dingli}[Crammer法则]
  $$
  A\mbox{非奇异}
  ~~\Longrightarrow~~
  \mbox{此方程组有唯一解,且}
  x_i = \frac{|A_i|}{|A|}, ~~ i = 1, \cd, n.
  $$
\end{dingli}

\pause
该结论非常漂亮,它把线性方程组的求解问题归结为计算$n+1$个$n$阶行列式的计算问题。

\end{frame}
%
%%%%%%%%%%%%%%%%%
\begin{frame}\ft{\secname}\fst{研究数值方法的必要性}

对于行列式的计算
\begin{dingli}[Laplace展开定理]
  $$
  |A| = a_{i1} A_{i1} + a_{i2} A_{i2} +  \cd + a_{in} A_{in}, \quad
  A_{ij}\mbox{为}a_{ij}\mbox{的代数余子式}
  $$      
\end{dingli}

\pause
该方法的运算量大的惊人,以至于完全不能用于实际计算。

\end{frame}
%
%%%%%%%%%%%%%%%%%
\begin{frame}\ft{\secname}

设$k$阶行列式所需乘法运算的次数为$m_k$,则
$$
m_k = k + k m_{k-1},
$$
于是有
$$
\begin{array}{ll}
  &m_n = n + n m_{n-1} \\[0.2cm]
  &= n + n[(n-1) + (n-1) m_{n-2}] \\[0.2cm]
  &= \cd \\[0.2cm]
  &= n + n(n-1) + n(n-1)(n-2) + \cd + n(n-1)\cd 3 \cdot 2\\[0.2cm]
  &> n!
\end{array}
$$

\end{frame}
%
%%%%%%%%%%%%%%%%%
\begin{frame}\ft{\secname}

  故用Crammer法则和Laplace展开定理求解一个$n$阶线性方程组,所需乘法运算的次数就大于
$$
(n+1)n! = (n+1)!.
$$

\end{frame}
%
%%%%%%%%%%%%%%%%%
\begin{frame}\ft{\secname}
在一台百亿次的计算机上求解一个25阶线性方程组,则至少需要
$$
\frac{26!}{10^{10}\times 3600 \times 24 \times 365}
\approx
\frac{4.0329\times 10^{28}}{3.1526\times 10^{17}}
\approx
13\mbox{亿年}
$$
\pause
而用下章介绍的消去法求解,则需要不到一秒钟。
\end{frame}
%
%
%
%%%%%%%%%%%%%%%%%
\begin{frame}\ft{\secname}%\fst{研究对象}
数值分析的研究对象为: 
\begin{itemize}
\item
线性代数\\
\item
曲线拟合\\
\item
数值逼近\\
\item
微积分\\
\item
微分方程\\
\item
积分方程\\
\item
$\cdots$
\end{itemize}
\end{frame}

%%%%%%%%%%%%%%%%%
\begin{frame}\ft{\secname}%\fst{研究任务}
数值分析的研究任务为: 
\begin{itemize}
\item
算法设计 
\item
理论分析
\item[]
\begin{itemize}
\item 算法的收敛性 
\item 稳定性 
\item 误差分析
\end{itemize} 

\item
复杂度分析
\item[]
\begin{itemize}
\item 时间复杂度 
\item 空间复杂度
\end{itemize} 
\end{itemize}

\end{frame}
%
%%%%%%%%%%%%%%%%%
\begin{frame}\ft{\secname}%\fst{特点}
数值分析的特点为: 
\begin{itemize}
\item
既有数学的抽象性与严格性,又有广泛的应用性;
\item
有自身的研究方法和理论系统;
\item
与计算机紧密结合,实用性很强。
\end{itemize}

\end{frame}
