\subsection{栈的应用}

\begin{frame}\ft{\subsecname}
由于栈“后进先出”的固有特性,故栈是程序设计中常用的工具和数据结构。
\end{frame}

\begin{frame}\ft{应用1:数制转换}
十进制整数$n$向其它进制数$d(=2, 8, 16)$的转换是计算机实现计算的基本问题。

\textcolor{acolor5}{转换法则:}
$$
n=(n\div d)\times d+ r
$$
其中$r$为余数,且$0\le r < d$。
\end{frame}

\begin{frame}\ft{应用1:数制转换}
\begin{li}
$(1348)_{10}=(2504)_8$,其运算过程如下:
\begin{table}
\begin{tabular}{ccc}\hline
$n$ & $n\div 8$ & $r$\\ \hline
1348 & 168 & 4\\
168  & 21  & 0\\
21   & 2   & 5\\
2    & 0   & 2\\\hline
\end{tabular}
\end{table}
\end{li}
\end{frame}
%
\begin{frame}[fragile,allowframebreaks]\ft{应用1:数制转换(程序)}
\lstinputlisting[
language=c,
]{Chapters/Ch03/Code/App/App1/Convert.c}
\end{frame}
%

\begin{frame}\ft{应用2:括号匹配}
在文字处理软件或编写程序时,常常需要检查一个字符串或一个表达式终端括号是否相匹配?
\vspace{0.1in}

\pause 
\textcolor{acolor5}{匹配思想:}
从左至右扫描一个字符串(或表达式),则\textcolor{acolor1}{每个右括号将于最近遇到的那个左括号相匹配}。可以在从左到右扫描过程中把所遇到的左括号存放到堆栈中,每当遇到一个右括号时,就将它与栈顶的左括号(如果存在)相匹配,同时从栈顶删除该左括号。  

\end{frame}


\begin{frame}\ft{应用2:括号匹配}
\textcolor{acolor5}{算法思想:}
设置一个栈,当读到左括号时,左括号进栈。当读到右括号时,则从栈中弹出一个元素,与读到的左括号进行匹配,若匹配成功,继续读入;否则匹配失败,返回{\tt FALSE}.    


\end{frame}

\begin{frame}[fragile,allowframebreaks]\ft{应用2:括号匹配(程序)}
\lstinputlisting[
language=C,
]{Chapters/Ch03/Code/App/App2/MatchBrackets.c}
\end{frame}

\begin{frame}\ft{前缀、中缀、后缀表达式}
它们都是对表达式的记法,其区别在于运算符相对操作数的位置不同:
\begin{itemize}
\item 前缀表达式:运算符位于其相关操作数的前面;
\item 中缀表达式:运算符位于其相关操作数的中间;
\item 后缀表达式:运算符位于其相关操作数的后面。
\end{itemize}
\end{frame}

\begin{frame}\ft{前缀、中缀、后缀表达式}
例如:
\begin{itemize}
\item 前缀表达式:$-~\times~+~3~4~5~6$
\item 中缀表达式:$(3~+~4)~\times ~5 ~- ~6$
\item 后缀表达式:$3~4~+~5~\times~6~-$
\end{itemize}
\end{frame}

\begin{frame}\ft{前缀、中缀、后缀表达式}
\textcolor{acolor3}{中缀表达式:}
人们常用的算术表示方法。

中缀表达式很容易被人的大脑理解和分析,但对计算机来说却是很复杂的。因此计算表达式的值时,通常需要将中缀表达式转换为前缀或后缀表达式,然后进行求值。

\textcolor{acolor5}{对计算机来说,计算前缀或后缀表达式的值非常简单。}
\end{frame}

\begin{frame}\ft{前缀、中缀、后缀表达式}
\textcolor{acolor3}{前缀表达式:} 也称“前缀记法、波兰式”,其运算符位于操作数之前。\vspace{0.1in}

\textcolor{acolor3}{后缀表达式:} 也称“后缀记法、逆波兰式”,其运算符位于操作数之后。
\end{frame}

\begin{frame}\ft{应用3:后缀表达式的计算机求值}
从左到右扫描表达式,
\begin{itemize}
\item[(1)] 遇到操作数,将数字入栈;
\item[(2)] 遇到运算符,弹出栈顶的两个数;
\item[(3)] 将这两个数用运算符做相应计算(次顶元素~op~栈顶元素),并将结果入栈;
\item[(4)] 重复(1)-(3)直到表达式右端,最后运算出的值即为表达式的结果。
\end{itemize}
\end{frame}

\begin{frame}\ft{应用3:后缀表达式的计算机求值}
\begin{li}
以后缀表达式
$$
3~4~+~5~\times~6~-
$$
为例,详解整个过程。
\end{li}
\end{frame}

\begin{frame}\ft{应用3:后缀表达式的计算机求值}
\begin{enumerate}
\item 将{\tt 3}和{\tt 4}入栈;
\item 遇到{\tt +},弹出{\tt 4}和{\tt 3},计算{\tt 3+4=7},再将{\tt 7}入栈;
\item 将{\tt 5}入栈;
\item 遇到{\tt *},弹出{\tt 5}和{\tt 7},计算{\tt 7*5=3},再将{\tt 35}入栈;
\item 将{\tt 6}入栈;
\item 遇到{\tt -},弹出{\tt 6}和{\tt 35},计算{\tt 35-6=29},即为最终结果。
\end{enumerate}
\end{frame}

\begin{frame}\ft{应用4:将中缀表达式转换为后缀表达式}
\begin{itemize}
\item[1] 初始化两个栈:运算符栈{\tt S1}和存储中间结果的栈{\tt S2};\\[0.1in]
\item[2] 从左到右扫描中缀表达式;\\[0.1in]
\item[3] 遇到操作数时,将其压入{\tt S2};
\end{itemize}
\end{frame}

\begin{frame}\ft{应用4:将中缀表达式转换为后缀表达式}
\begin{itemize}
\item[4] 遇到运算符,比较其与{\tt S1}栈顶运算符的优先级:\\[0.1in]
\begin{itemize}
\item[4.1] 若{\tt S1}为空,或栈顶运算符为“(”,则压入{\tt S1};\\[0.1in]
\item[4.2] 否则,若其优先级高于栈顶运算符,则压入{\tt S1};\\[0.1in]
\item[4.3] 否则,将{\tt S1}的栈顶运算符弹出并压入到{\tt S2}中,再次转到4.1与{\tt S1}中新的栈顶运算符做比较;
\end{itemize}
\end{itemize}
\end{frame}

\begin{frame}\ft{应用4:将中缀表达式转换为后缀表达式}
\begin{itemize}
\item[5] 遇到括号时,\\[0.1in]
\begin{itemize}
\item[5.1] 如果是“(”,则压入{\tt S1};\\[0.1in]
\item[5.2] 如果是“)”,则依次弹出{\tt S1}的栈顶元素,并压入S2,直到遇到“(”为止,此时这一对括号丢弃;
\end{itemize}
\end{itemize}
\end{frame}

\begin{frame}\ft{应用4:将中缀表达式转换为后缀表达式}
\begin{itemize}
\item[6] 重复步骤2-5,直到表达式的最右端;\\[0.1in]
\item[7] 将{\tt S1}中剩余的运算符依次弹出并压入S2;\\[0.1in]
\item[8] 依次弹出{\tt S2}中的元素并输出,结果的逆序即为对应的后缀表达式。
\end{itemize}
\end{frame}



\begin{frame}\ft{应用4:将中缀表达式转换为后缀表达式}
\begin{scriptsize}
\begin{table}
\centering
\caption{{\tt 中缀表达式 1 + ((2 + 3) + 4) * 5}}
\begin{tabular}{|c|l|l|}\hline
{\tt 扫描到的字符} & {\tt S2(栈底->栈顶)} & {\tt S1(栈底->栈顶)} \\\hline
{\tt 1} &   {\tt 1} &         \\\hline 
{\tt +} &   {\tt 1} &  {\tt +}\\\hline 
{\tt (} &   {\tt 1} &  {\tt +(}\\\hline 
{\tt (} &   {\tt 1} &  {\tt +((}\\\hline 
{\tt 2} &   {\tt 12} &  {\tt +((}\\\hline 
{\tt +} &   {\tt 12} &  {\tt +((+}\\\hline 
{\tt 3} &   {\tt 123} &  {\tt +((+}\\\hline 
{\tt )} &   {\tt 123+} &  {\tt +(}\\\hline 
{\tt +} &   {\tt 123+} &  {\tt +(+}\\\hline 
{\tt 4} &   {\tt 123+4} &  {\tt +(+}\\\hline 
{\tt )} &   {\tt 123+4+} &  {\tt +}\\\hline 
{\tt *} &   {\tt 123+4} &  {\tt +*}\\\hline 
{\tt 5} &   {\tt 123+45} &  {\tt +*}\\\hline 
        &   {\tt 123+45*+} &  \\\hline 
\end{tabular}
\end{table}
\end{scriptsize}
\end{frame}


\begin{frame}\ft{应用4:将中缀表达式转换为后缀表达式}
\begin{scriptsize}
\begin{table}
\centering
\caption{{\tt 中缀表达式 1 + ((2 + 3) * 4) - 5}}
\begin{tabular}{|c|l|l|}\hline
{\tt 扫描到的字符} & {\tt S2(栈底->栈顶)} & {\tt S1(栈底->栈顶)} \\\hline
{\tt 1} &   {\tt 1} &         \\\hline 
{\tt +} &   {\tt 1} &  {\tt +}\\\hline 
{\tt (} &   {\tt 1} &  {\tt +(}\\\hline 
{\tt (} &   {\tt 1} &  {\tt +((}\\\hline 
{\tt 2} &   {\tt 12} &  {\tt +((}\\\hline 
{\tt +} &   {\tt 12} &  {\tt +((+}\\\hline 
{\tt 3} &   {\tt 123} &  {\tt +((+}\\\hline 
{\tt )} &   {\tt 123+} &  {\tt +(}\\\hline 
{\tt *} &   {\tt 123+} &  {\tt +(*}\\\hline 
{\tt 4} &   {\tt 123+4} &  {\tt +(+}\\\hline 
{\tt )} &   {\tt 123+4*} &  {\tt +}\\\hline 
{\tt -} &   {\tt 123+4+} &  {\tt -}\\\hline 
{\tt 5} &   {\tt 123+4+5} &  {\tt -}\\\hline 
        &   {\tt 123+4+5-} &  \\\hline 
\end{tabular}
\end{table}
\end{scriptsize}
\end{frame}




