\subsection{链栈}
\begin{frame}[fragile]\ft{\subsecname}
\begin{dingyi}
栈的链式存储结构称为链栈,是操作受限的单链表,其插入和删除操作只能在表头位置上进行。
\end{dingyi}
\end{frame}

\begin{frame}[fragile]\ft{\subsecname}
\begin{figure}
\centering
\begin{tikzpicture}
\tikzstyle{information text}=[rounded corners,fill=blue!10,inner sep=1ex]
\def\x{1}
\def\y{0.6}

\foreach \r in {0,4}{
\ifthenelse{\r=4}{
\foreach \c in {6,4,2,0}{
\draw[thick,fill=red!10,rounded corners](\r*\x,\c*\y)rectangle(\r*\x+1.5*\x,\c*\y+\y);
\draw[thick] (\r*\x+\x,\c*\y)--(\r*\x+\x,\c*\y+\y);
\ifthenelse{6=\c}{\node[]at(\r*\x+0.5*\x,\c*\y+0.5*\y){$a_4$};}{}
\ifthenelse{4=\c}{\node[]at(\r*\x+0.5*\x,\c*\y+0.5*\y){$a_3$};}{}
\ifthenelse{2=\c}{\node[]at(\r*\x+0.5*\x,\c*\y+0.5*\y){$a_2$};}{}
\ifthenelse{0=\c}{
\node[]at(\r*\x+0.5*\x,\c*\y+0.5*\y){$a_1$};            
}{}

\ifthenelse{6=\c}{
\draw[->,thick] (\r*\x-0.6*\x,\c*\y+0.5*\y)node[left]{{\tt top}} --(\r*\x-0.1*\x,\c*\y+0.5*\y);
}{}

\draw[->,thick] (\r*\x+1.25*\x,\c*\y+0.5*\y) --(\r*\x+1.25*\x,\c*\y-0.9*\y);

\ifthenelse{\c=0} {
\filldraw[thick] (\r*\x+\x,\c*\y-1.5*\y) rectangle (\r*\x+1.5*\x,\c*\y-\y);
\node[below] at (\r*\x+0.75*\x,\c*\y-2*\y) {非空栈};
}

}
}
%{\foreach \c in {6}{
%\draw[thick,fill=red!10,rounded corners](\r*\x,\c*\y)rectangle(\r*\x+1.5*\x,\c*\y+\y);
%\draw[thick] (\r*\x+\x,\c*\y)--(\r*\x+\x,\c*\y+\y);
%\draw[->,>=stealth] (\r*\x-0.5*\x,\c*\y+0.5*\y)node[left]{{\tt top}} --(\r*\x,\c*\y+0.5*\y);
%
%\draw[->,thick] (\r*\x+1.25*\x,\c*\y+0.5*\y) --(\r*\x+1.25*\x,\c*\y-0.9*\y);
%\filldraw[thick] (\r*\x+\x,\c*\y-1.5*\y) rectangle (\r*\x+1.5*\x,\c*\y-\y);
%\node[below] at (\r*\x+0.75*\x,\c*\y-2*\y) {空栈};
%}
%}
}
\end{tikzpicture}
\end{figure}

\end{frame}
%
%
\begin{frame}[fragile]\ft{\subsecname}
\begin{itemize}
\item 对于链栈,通常不需要头结点;\\[0.1in]
\item 对于链栈,基本不存在栈满的情况;\\[0.1in]
\item 对于空栈,通常指的是{\tt top == NULL}.
\end{itemize}
\end{frame}
%




\begin{frame}
\begin{center}
\textcolor{acolor5}{\Large 链栈之完整程序}
\end{center}
\end{frame}
%




\begin{frame}[fragile,allowframebreaks]\ft{{\tt linkstack.h}}
\lstinputlisting[
language=C,
]{Chapters/Ch03/Code/LinkStack/LinkStack.h}
\end{frame}



\begin{frame}[fragile]\ft{{\tt Init.c}}
\lstinputlisting[
language=C,
]{Chapters/Ch03/Code/LinkStack/Init.c}
\end{frame}
%
\begin{frame}[fragile]\ft{{\tt Push.c}}
\lstinputlisting[
language=C,
]{Chapters/Ch03/Code/LinkStack/Push.c}
\end{frame}
%
\begin{frame}[fragile]\ft{{\tt Pop.c}}
\lstinputlisting[
language=C,
]{Chapters/Ch03/Code/LinkStack/Pop.c}
\end{frame}
%
\begin{frame}[fragile]\ft{{\tt PrintS.c}}
\lstinputlisting[
language=C,
]{Chapters/Ch03/Code/LinkStack/PrintS.c}
\end{frame}
%
\begin{frame}[fragile]\ft{{\tt main.c}}
\lstinputlisting[
language=C,
]{Chapters/Ch03/Code/LinkStack/main.c}
\end{frame}
