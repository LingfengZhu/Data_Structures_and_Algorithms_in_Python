\subsection{基本概念}

\begin{frame}\ft{\subsecname}


线性表是一种典型的线性结构,其中的数据元素是{有序且有限}, 并且
\begin{itemize}
\item
有唯一首元;
\item
有唯一末元;
\item
除首元外,每个元素均有唯一的直接前驱;
\item
除末元外,每个元素均有唯一的直接后继. 
\end{itemize}•

\end{frame}

\begin{frame}\ft{\subsecname} 
\begin{dingyi}[线性表]
由$n$个数据元素$a_1,a_2,\cd,a_n$组成的有限序列($n\ge 0$, 数据元素又称结点).
\begin{itemize}
\item $a_i$的数据类型相同;
\item $n$称为线性表的长度.
\end{itemize}
\end{dingyi}

\end{frame}

\begin{frame}\ft{\subsecname} 
 
\begin{enumerate}
\item $n=0$为空表;\\[0.1in]
\item $n>0$为非空的线性表, 记为$(a_1,a_2,\cd,a_n)$. 
\begin{itemize}
\item
$a_1$称为首结点, $a_n$称为尾结点。 
\item 
$a_1,a_2,\cd,a_{i-1}$都是$a_i%(2\le i \le n)
$的前驱, 其中$a_{i-1}$是$a_i$的直接前驱。
\item 
$a_{i+1},a_{i+2},\cd,a_{n}$都是$a_i%(1\le i \le n-1)
$的后继, 其中$a_{i+1}$是$a_i$的直接后继. 
\end{itemize}
\end{enumerate}
\end{frame}

\begin{frame}\ft{\subsecname} 

线性表中的数据元素$a_i$所代表的具体含义随具体应用的不同而不同,
只不过是一个抽象的表示符号。

\begin{itemize}
\item 
结点可以是单值元素(每个元素只有一个数据项)
\end{itemize}

\begin{li}
字母表:(A,~B,~C,~$\cd$,~Z)
\end{li}	

\begin{li}
扑克点数:(2,~3,~4,~$\cd$,~J,~Q,~K,~A)
\end{li}	

\end{frame}

\begin{frame}\ft{\subsecname}\fst{逻辑结构}
\begin{itemize}
\item 
结点可以是记录型元素。
\begin{itemize} 
\item 每个元素可含多个数据项, 每一项称为结点的一个域。
每个元素有一个可以唯一标识每个结点的域,,称为{关键字(key word)}。
\end{itemize}
\end{itemize}
\pause 
\begin{table}
\centering
\caption{某班2014级同学的基本情况}
\begin{tabular}{cccc}\hline
学号&姓名&性别&出生日期\\\hline
$20140212001$&张强&男&$06/24/1992$\\
$20140212002$&王明&男&$08/22/1992$\\
$\cd$&$\cd$&$\cd$&$\cd$\\
$20140212030$&李娟&女&$09/12/1992$\\\hline
\end{tabular}  
\end{table}

\end{frame}

\begin{frame}\ft{\subsecname}\fst{逻辑结构}
\begin{itemize}
\item 
若结点按值从小到大(或从大到小)排列, 
称线性表是{有序}的。
\item 
线性表的长度可根据需要增长或缩短。
\item 
可对结点进行访问、插入和删除操作。
\end{itemize}

\end{frame}


\begin{frame}[fragile]\ft{\subsecname}\fst{线性表的抽象数据类型定义}
\begin{lstlisting}[mathescape=true,frame=tb]
  ADT List{
    Data:
      `数据对象集合为$(a_1,a_2,\cd,a_n)$, 各元素类型均为DataType. 其中,除首元素外,每个元素有且仅有一个直接前驱,除最后一个元素外,每个元素有且仅有一个直接后继. 数据元素之间为一对一的关系.`
    Operation:
      InitList(&L):    `初始化操作,建立一个空的线性表L.`
      ListEmpty(L):    `若线性表为空,返回true,否则返回false.`
      ClearList(&L):   `将线性表清空.`
      GetElem(L,i,&e): `将线性表L中的第i个位置的元素返回给e.`
      LocateElem(L,e):  `在线性表L中查找与给定值e相等的元素`
      ListInsert(&L,i,&e): `在线性表L中的第i个位置插入新元素e.`
      ListDelete(&L,i,&e):  
           `删除线性表L中第i个位置的元素,并用e返回该值.`
      ListLength(L):      `返回线性表L的元素个数.`      
      ...
  } ADT List
\end{lstlisting}

\end{frame}



\begin{frame}\ft{\subsecname}\fst{线性表的抽象数据类型定义}
\begin{li}
将两个线性表A和B合并,即把B中存在而A中不存在的数据元素插入到A中。
\end{li}
\pause 
\begin{lstlisting}[frame=tb,backgroundcolor=\color{red!10}]
void union(List *La,List Lb){
  int La_len,Lb_len,i;
  ElemType e;
  La_Len=ListLength(La);
  Lb_Len=ListLength(Lb);
  for(i=1;i<=Lb_len;i++){
    GetElem(Lb,i,e);
    if(!LocateElem(La,e))
      ListInsert(La,++La_len,e);
  }
} 
\end{lstlisting}

\end{frame}
