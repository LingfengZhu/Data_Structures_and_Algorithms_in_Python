% XeLaTeX can use any Mac OS X font. See the setromanfont command below.
% Input to XeLaTeX is full Unicode, so Unicode characters can be typed directly into the source.

% The next lines tell TeXShop to typeset with xelatex, and to open and save the source with Unicode encoding.

%!TEX TS-program = xelatex
%!TEX encoding = UTF-8 Unicode

\documentclass[10pt]{article}
\usepackage{geometry}                % See geometry.pdf to learn the layout options. There are lots.
\geometry{letterpaper}                   % ... or a4paper or a5paper or ... 
%\geometry{landscape}                % Activate for for rotated page geometry
%\usepackage[parfill]{parskip}    % Activate to begin paragraphs with an empty line rather than an indent
\usepackage{graphicx}
\usepackage{amssymb}
%%%% Package
\usepackage{amsmath,amssymb,amsthm}
\usepackage[UTF8,noindent]{ctex}
\usepackage{pgf}
\usepackage{tikz}
\usetikzlibrary{calc}
\usetikzlibrary{arrows,snakes,backgrounds,shapes,shadows}
\usetikzlibrary{matrix,fit,positioning,decorations.pathmorphing}
\usepackage{listings}
\lstset{
  keywordstyle=\color{blue!70},
  frame=single,
  basicstyle=\ttfamily,
  commentstyle=\small\color{red},
  breakindent=0pt,
  rulesepcolor=\color{red!20!green!20!blue!20},
  rulecolor=\color{black},
  tabsize=4,
  numbersep=5pt,
  breaklines=true,
  %% backgroundcolor=\color{red!10},
  showspaces=false,
  showtabs=false,
  extendedchars=false,
  escapeinside=``,
  frame=no,
}

%%%% New Commands
\newcommand{\blue}{\textcolor{blue}}
\newcommand{\red}{\textcolor{red}}
\newcommand{\purple}{\textcolor{electricpurple}}
\newcommand{\ds}{\displaystyle}
\newcommand{\cd}{\cdots}
\newcommand{\dd}{\ddots}
\newcommand{\vd}{\vdots}
\newcommand{\id}{\iddots}


% Will Robertson's fontspec.sty can be used to simplify font choices.
% To experiment, open /Applications/Font Book to examine the fonts provided on Mac OS X,
% and change "Hoefler Text" to any of these choices.

\usepackage{fontspec,xltxtra,xunicode}
\defaultfontfeatures{Mapping=tex-text}
\setromanfont[Mapping=tex-text]{Hoefler Text}
\setsansfont[Scale=MatchLowercase,Mapping=tex-text]{Gill Sans}
\setmonofont[Scale=MatchLowercase]{Andale Mono}

\begin{document}
\renewcommand{\proofname}{\textbf{证明}}
%%%% New Theorem
\newtheorem{li}{例}
\newtheorem{jielun}{结论}
\newtheorem{dingli}{定理}
\newtheorem{mingti}{{命题}} 
\newtheorem{yinli}{{引理}} 
\newtheorem{tuilun}{{推论}}
\newtheorem{dingyi}{{定义}} 
\newtheorem*{jie}{{解}}
\newtheorem*{zhengming}{{证明}}
\newtheorem{zhu}{{注}}
\newtheorem*{zhu*}{{注}}
\newtheorem{xingzhi}{{性质}}
\newtheorem{wenti}{{问题}}
\newtheorem{xiti}{{习题}}

\title{图的表示}
\author{张晓平}
%\date{}                                           % Activate to display a given date or no date
\maketitle

\section{邻接矩阵}
%由于图是有顶点和边或弧两部分组成的,可以考虑分两个结构分别来存储。
%\begin{itemize}
%\item[(1)] 顶点不分大小、主次,可用一个一维数组来存储。
%\item[(2)] 边或弧为顶点与顶点之间的关系,可用一个二维数组来存储。
%\end{itemize}
%此即邻接矩阵的存储方案。


图的邻接矩阵(Adjacency matrix)存储方式是用两个数组来表示图。一个一维数组存储图中顶点信息,一个二维数组(称为邻接矩阵)存储图中的边或弧的信息。

设图$G$有$n$个顶点,则邻接矩阵为一个$n\times n$的方阵,定义为
$$
e[i][j]=\left\{
\begin{array}{ll}
1,& \mbox{ if $(v_i,v_j)\in E$ or $\langle v_i,v_j\rangle\in E$}, \\[0.05in]
0,& \mbox{ otherwise}.
\end{array}
\right.
$$


\tikzstyle{state}=[circle,draw=blue!50,fill=blue!20,thick,inner sep=0pt,minimum size=6mm]

\begin{figure}
\centering
\begin{tikzpicture}[scale=1.5,node distance=3cm,semithick,inner sep=1pt,bend angle=45,auto]
%\draw[help lines] (-1,-1) grid (5,3);
\node[state]   (v0) at (30:1)  {$v_0$};
\node[state]   (v1) at (0:0)   {$v_1$};
\node[state]   (v2) at (-40:1) {$v_2$};
\node[state]   (v3) at (-5:1.8)  {$v_3$};
\path
(v0) edge (v1)
     edge (v2)   
     edge (v3)
(v1) edge (v2)
(v2) edge (v3);   

\matrix (am) at (3,0) [matrix of math nodes,left delimiter=(,right delimiter=),row sep=10pt,column sep=10pt,right] 
  {
    0 & 1 & 1 & 1\\
    1 & 0 & 1 & 0\\
    1 & 1 & 0 & 1\\
    1 & 0 & 1 & 0\\
  };
  \node at (am-1-1) [above=10pt] {$v_0$}; 
  \node at (am-1-2) [above=10pt] {$v_1$};       
  \node at (am-1-3) [above=10pt] {$v_2$};       
  \node at (am-1-4) [above=10pt] {$v_3$};
  \node at (am-1-1) [left=15pt] {$v_0$}; 
  \node at (am-2-1) [left=15pt] {$v_1$};       
  \node at (am-3-1) [left=15pt] {$v_2$};       
  \node at (am-4-1) [left=15pt] {$v_3$};             

%\matrix (v) at (3,2) [matrix of math nodes,left delimiter=(,right delimiter=),row sep=10pt,column sep=10pt,right]{
%  v_0 & v_1 & v_2 & v_3\\
%};


\end{tikzpicture}

\caption{无向图的邻接矩阵为对称矩阵}
\end{figure}

有了邻接矩阵,可以很容易地得到无向图中的信息:
\begin{itemize}
\item[1] 可以很容易地判定任意两顶点是否有边。 
\item[2] 顶点$v_i$的度等于邻接矩阵中第$i$行的元素之和。
\item[3] 欲求顶点$v_i$的所有邻接点,只需扫描邻接矩阵的第$i$行,若$e[i][j]=1$,则$v_j$为其邻接点。
\end{itemize}


\begin{figure}[htbp]
\centering
\begin{tikzpicture}[scale=1.5,->,>=stealth',node distance=3cm,semithick,inner sep=1pt,bend angle=45,auto]
%\draw[help lines] (-1,-1) grid (5,3);
\node[state]   (v0) at (30:1)  {$v_0$};
\node[state]   (v1) at (0:0)   {$v_1$};
\node[state]   (v2) at (-40:1) {$v_2$};
\node[state]   (v3) at (-5:1.8)  {$v_3$};
\path
(v0) edge (v3)
(v1) edge (v0)
     edge[bend right=10] (v2)   
(v2) edge[bend right=10] (v1)
     edge (v0);   

\matrix (am) at (3,0) [matrix of math nodes,left delimiter=(,right delimiter=),row sep=10pt,column sep=10pt,right] 
  {
    0 & 0 & 0 & 1\\
    1 & 0 & 1 & 0\\
    1 & 1 & 0 & 0\\
    0 & 0 & 0 & 0\\
  };
  \node at (am-1-1) [above=10pt] {$v_0$}; 
  \node at (am-1-2) [above=10pt] {$v_1$};       
  \node at (am-1-3) [above=10pt] {$v_2$};       
  \node at (am-1-4) [above=10pt] {$v_3$};
  \node at (am-1-1) [left=15pt] {$v_0$}; 
  \node at (am-2-1) [left=15pt] {$v_1$};       
  \node at (am-3-1) [left=15pt] {$v_2$};       
  \node at (am-4-1) [left=15pt] {$v_3$};             

\matrix (v) at (3,2) [matrix of math nodes,left delimiter=(,right delimiter=),row sep=10pt,column sep=10pt,right]{
  v_0 & v_1 & v_2 & v_3\\
};


\end{tikzpicture}

\caption{有向图的邻接矩阵不是对称矩阵}
\end{figure}

有了邻接矩阵,可以很容易地得到图中的信息。
\begin{itemize}
\item[1] 顶点$v_i$的入度等于邻接矩阵中第$i$列的元素之和;顶点$v_i$的出度等于邻接矩阵中第$i$行的元素之和。
\item[2] 欲判断顶点$v_i$到$v_j$是否有弧,只需查找$e[i][j]$是否为1。
\item[3] 欲求顶点$v_i$的所有邻接点,只需扫描邻接矩阵的第$i$行,若$e[i][j]=1$,则$v_j$为其邻接点。
\end{itemize}
\end{document}  