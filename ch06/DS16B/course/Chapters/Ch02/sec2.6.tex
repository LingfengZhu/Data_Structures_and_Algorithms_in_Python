\section{一元多项式的表示和相加}

\begin{frame}\ft{一元多项式的表示}
一元多项式
$$
p(x)=p_0+p_1x+p_2x^2+\cd+p_nx^n
$$
由$n+1$个系数唯一确定,在计算机中可用线性表
$$
(p_0,p_1,p_2,\cd,p_n)
$$
表示。既然是线性表,就可以用顺序表和链表来实现。
\end{frame}

\begin{frame}[fragile]\ft{一元多项式的表示}
\begin{block}{顺序存储表示的类型}
\begin{lstlisting}[language=C,frame=none]
typedef struct{
  float coef;
  int expn;
} ElemType;
\end{lstlisting}
\end{block}

\begin{block}{链式存储表示的类型}
\begin{lstlisting}[language=C,frame=none]
typedef struct poly{
  float coef;
  int expn;
  struct poly *next;
} Poly;
\end{lstlisting}
\end{block}
\end{frame}

\begin{frame}[fragile]\ft{一元多项式的相加}
不失一般性,设有两个一元多项式:
$$
\begin{aligned}
&p(x)=p_0+p_1x+p_2x^2+\cd+p_nx^n\\
&q(x)=q_0+q_1x+q_2x^2+\cd+q_mx^m(m<n)
\end{aligned}
$$
相加得
$$
r(x)=p(x)+q(x),
$$
它由线性表
$$
((p_0+q_0),(p_1+q_1),(p_2+q_2),\cd,(p_m+q_m),\cd,p_n)
$$
唯一表示。
\end{frame}

\begin{frame}[fragile]\ft{一元多项式的相加}
\begin{itemize}
\item[(1)] 顺序存储表示的相加
\begin{block}{顺序表的定义}
\begin{lstlisting}[language=C,frame=none]
typedef struct {
  ElemType a[MAX_SIZE];
  int length;
} Sqlist;
\end{lstlisting}
\end{block}
用顺序表示的相加非常简单。访问第五项可直接访问:
\begin{lstlisting}[language=C,frame=none]
Sqlist L;
L.a[4].coef;
L.a[4].expn;
\end{lstlisting}
\end{itemize}
\end{frame}

\begin{frame}[fragile]\ft{一元多项式的相加}
\begin{itemize}
\item[(2)] 链式存储表示的相加
\item[]当采用链式存储表示时,根据结点类型定义,凡是系数为零的项不在链表中出现,从而可以大大减少链表的长度。\\[0.1in]
\item[]
一项多项式相加的实质是
\item[$\diamond$]
指数不同:是链表的合并。
\item[$\diamond$]
指数相同:系数相加,和为0,去掉结点,和不为0,修改结点的系数域。
\end{itemize}
\end{frame}

\begin{frame}[fragile]\ft{一元多项式的相加}
\begin{block}{算法1}
在原来两个多项式链表的基础上进行相加,相加后原来两个多项式链表不再存在。当然再要对原来两个多项式进行其它操作就不允许了。
\end{block}
\end{frame}

\begin{frame}[fragile,allowframebreaks]\ft{算法1-}
\begin{lstlisting}[language=C,frame=none,extendedchars=false]
Poly *add_poly(Poly *a, Poly *b){
  Poly *Lc,*pc,*pa,*pb,*ptr; float x;
  Lc=pc=La; pa=La->next; pb=Lb_next;
  while(pa!=NULL&&pb!=NULL){
    if (pa->expn<pb->expn){
      pc->next=pa; pc=pa; pa=pa->next;
    } else if (pa->expn>pb->expn){
      pc->next=pb; pc=pb; pb=pb->next;
    } else {
      x=pa->coef+pb->coef;
      if(abs(x)<=1.e-6){
        ptr=pa; pa=pa->next; free(ptr);
        ptr=pb; pb=pb->next; free(ptr);
      } else {
        pc->next=pa; pa->coef=x; pc=pa; pa=pa->next;
        ptr=pb; pb=pb->next; free(pb);
      }
    }
  }
  if (pa==NULL) pc->next=pb;
  else pc->next=pa;
  return (Lc);
}
\end{lstlisting}
\end{frame}	

\begin{frame}[fragile]\ft{一元多项式的相加}
\begin{block}{算法2}
对两个多项式链表进行相加,生成一个新的相加后的结果多项式链表,
原来两个多项式仍然存在,不发生任何改变。如果要再对原来两个多项式进行其它操作也不影响。
\end{block}
\end{frame}

\begin{frame}[fragile,allowframebreaks]\ft{算法2}
\begin{lstlisting}[language=C,frame=none,extendedchars=false]
Poly *add_poly(Poly *a, Poly *b){
  Poly *Lc,*pc,*pa,*pb,*p; float x;
  Lc=pc=(Poly *)malloc(sizeof(Poly)); 
  pa=La->next; pb=Lb_next;
  while(pa!=NULL&&pb!=NULL){
    if (pa->expn<pb->expn){
      p=(Poly *)malloc(sizeof(Poly)); 
      p->coef=pa->coef; p->expn=pa->expn;
      p->next=NULL;
      pc->next=p; pc=p; pa=pa->next;
    } else if (pa->expn>pb->expn){
      p=(Poly *)malloc(sizeof(Poly)); 
      p->coef=pb->coef; p->expn=pb->expn;
      p->next=NULL;
      pc->next=p; pc=p; pb=pb->next;
    } 
    if (pa->expn==pb->expn){
      x=pa->coef+pb->coef;
      if(abs(x)<=1.e-6){
        pa=pa->next;  pb=pb->next;  
      } else {
        p=(Poly *)malloc(sizeof(Poly));
        p->coef=x; p->expn=pb->expn;
        p->next=NULL;
        pc->next=pa; pc=p; pa=pa->next;
        pb=pb->next; free(pb);
      }
    }
  }
  if (pb!=NULL)  
    while (pb!=NULL) {
      p=(Poly*)malloc(sizeof(Poly));
      p->coef=pb->coef; p->expn=pb->expn;
      p->next=NULL;
      pc->next=p; pc=p; pb=pb->next;
    }
  if (pa!=NULL)  
    while (pa!=NULL) {
      p=(Poly*)malloc(sizeof(Poly));
      p->coef=pa->coef; p->expn=pa->expn;
      p->next=NULL;
      pc->next=p; pc=p; pa=pa->next;
    }
}
\end{lstlisting}
\end{frame}	