\begin{figure}  
  \centering
  \begin{tikzpicture}[scale=1.5]
    \tikzstyle{state}=[draw,circle]

        % The graphic
    \node at (-.5,0)[left,fill=blue!20,draw,starburst,drop shadow,text=red,text centered]
    (1){
      {物理结构}
    } ;
    \node at (0.5,0)[right,fill=blue!20,draw,rectangle,rounded corners=3mm,text=blue,text width=6.5cm]
    (2){
      数据的逻辑结构在计算机中的存储方式.
    }; 
    \pause
    \node at (2.5,-1.5) [fill=red!20,draw,ellipse callout, callout relative pointer={(-1.,0.4)},text width=5.5cm]{\small
      数据的存储结构应正确反映数据元素之间的逻辑关系。如何存储数据元素之间的逻辑关系,是实现物理结构的重点和难点。
    };
    \pause 
    \node at (-1.5,-1.8) [text width=1.8cm,decorate,decoration=saw,fill=green!20,draw,circle,text width=3cm]{
    \begin{itemize}
    \item 顺序存储\\
    \item 链式存储
    \end{itemize}
    };

    
  \end{tikzpicture}
\end{figure}


