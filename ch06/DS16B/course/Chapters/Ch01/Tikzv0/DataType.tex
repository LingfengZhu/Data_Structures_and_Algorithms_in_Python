\documentclass{article}
\usepackage{CJK} 
\usepackage{graphics}
\usepackage{pgf}
\usepackage{tikz}
\usetikzlibrary{calc,shadows}
\usetikzlibrary{decorations.markings,scopes}
\usetikzlibrary{arrows,snakes,backgrounds,shapes}
\usetikzlibrary{decorations.pathmorphing}
\newcommand{\blue}{\textcolor{blue}}
\newcommand{\red}{\textcolor{red}}
\newcommand{\purple}{\textcolor{purple}}


\pgfrealjobname{survey}
\begin{document}
\begin{CJK}{UTF8}{gkai} 
  \beginpgfgraphicnamed{DataType}
  \begin{tikzpicture}
    % The graphic
    \node at (0,0)[fill=blue!20,draw,starburst,drop shadow,text width=6cm]{
      \blue{\large 数据类型}\\
      指一组性质相同的值的集合及定义在此集合上的一些操作的总称。
    };

    \node at (0,-4) [fill=blue!20,draw,ellipse,text width=7cm]
    { 
      在C语言中,按照取值的不同,数据类型可分为:
      \begin{itemize}
      \item 原子类型:是不可再分解的基本类型,包括整型、浮点型、字符型等;
      \item 由若干个类型组合而成,是可以再分解。如整型数组由若干个整型数据组成。
      \end{itemize}
    };

  \end{tikzpicture}
  \endpgfgraphicnamed  
\end{CJK}

\end{document}


