\documentclass{article}
\usepackage{CJK} 
\usepackage{graphics}
\usepackage{pgf}
\usepackage{tikz}
\usetikzlibrary{calc,shadows}
\usetikzlibrary{decorations.markings,scopes}
\usetikzlibrary{arrows,snakes,backgrounds,shapes}
\usetikzlibrary{decorations.pathmorphing}
\usepackage{listings}
\renewcommand{\ttdefault}{pcr}
\lstset{
  keywordstyle=\color{blue!70},
  frame=single,
  basicstyle=\ttfamily\bfseries\small,
  commentstyle=\small\color{red},
  rulesepcolor=\color{red!20!green!20!blue!20},
  tabsize=4,
  numbersep=5pt,
  %% backgroundcolor=\color{black!10},
  showspaces=false,
  showtabs=false,
  extendedchars=false,
  escapeinside=``,
  frame=no
}

\newcommand{\blue}{\textcolor{blue}}
\newcommand{\red}{\textcolor{red}}
\newcommand{\purple}{\textcolor{purple}}


\pgfrealjobname{survey}
\begin{document}
\begin{CJK}{UTF8}{gkai} 
  \beginpgfgraphicnamed{AlgEffBefore}
  \begin{tikzpicture}[node distance=4.3cm]
  \node [fill=blue!20,draw,starburst,drop shadow,text width=6cm] (1) {
  \blue{\large 事前分析估算方法}\\
  在编制程序前,依据统计方法对算法进行估算。
  };


\node[below left of=1,rectangle split,rectangle split parts=5,rounded corners=3mm,draw,text width=7cm,fill=red!20] (2){
\blue{程序运行时间取决于}
      \nodepart{two}
      (1)~算法采用的策略、方法 (\red{算法好坏的根本})
      \nodepart{three}
      (2)~编译产生的代码质量 (\red{软件支持})
      \nodepart{four}
      (3)~问题的输入规模 
      \nodepart{five}
      (4)~机器执行指令的速度 (\red{硬件性能})
    };
    \node[below right of=1,rectangle split,ellipse,draw,text width=2.5cm,fill=red!20] {
      程序的运行时间,依赖于算法的好坏和问题的输入规模。
    };

  \end{tikzpicture}
  \endpgfgraphicnamed  
\end{CJK}

\end{document}


