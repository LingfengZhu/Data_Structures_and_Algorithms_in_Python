\documentclass{article}
\usepackage{amsmath,amssymb,amsfonts}
\usepackage{CJK} 
\usepackage{graphics}
\usepackage{pgf}
\usepackage{tikz}
\usetikzlibrary{calc,shadows}
\usetikzlibrary{decorations.markings,scopes}
\usetikzlibrary{arrows,snakes,backgrounds,shapes}
\usetikzlibrary{decorations.pathmorphing}
\usepackage{listings}
\renewcommand{\ttdefault}{pcr}
\lstset{
  keywordstyle=\color{blue!70},
  frame=single,
  basicstyle=\ttfamily\bfseries\small,
  commentstyle=\small\color{red},
  rulesepcolor=\color{red!20!green!20!blue!20},
  tabsize=4,
  numbersep=5pt,
  %% backgroundcolor=\color{black!10},
  showspaces=false,
  showtabs=false,
  extendedchars=false,
  escapeinside=``,
  frame=no
}

\newcommand{\blue}{\textcolor{blue}}
\newcommand{\red}{\textcolor{red}}
\newcommand{\purple}{\textcolor{purple}}


\pgfrealjobname{survey}
\begin{document}
\begin{CJK}{UTF8}{gkai} 
  \beginpgfgraphicnamed{FuncAymInc}
  \begin{tikzpicture}[node distance=4.3cm]
  \node [fill=blue!20,draw,starburst,drop shadow,text width=6cm] (1) {
  \blue{\large 函数的渐近增长}\\
  给定两个函数$f(n)$和$g(n)$,若$\exists N \in \mathbb N$, s.t. $\forall n>N$, $f(n)$总是比$g(n)$大,
  我们就说$f(n)$的增长渐近快于$g(n)$.
  };
  \end{tikzpicture}
  \endpgfgraphicnamed  
\end{CJK}

\end{document}


