\documentclass{article}
\usepackage{CJK} 
\usepackage{graphics}
\usepackage{pgf}
\usepackage{tikz}
\usetikzlibrary{calc,shadows}
\usetikzlibrary{decorations.markings,scopes}
\usetikzlibrary{arrows,snakes,backgrounds,shapes}
\usetikzlibrary{decorations.pathmorphing}
\usepackage{listings}
\renewcommand{\ttdefault}{pcr}
\lstset{
  keywordstyle=\color{blue!70},
  frame=single,
  basicstyle=\ttfamily\bfseries\small,
  commentstyle=\small\color{red},
  rulesepcolor=\color{red!20!green!20!blue!20},
  tabsize=4,
  numbersep=5pt,
  %% backgroundcolor=\color{black!10},
  showspaces=false,
  showtabs=false,
  extendedchars=false,
  escapeinside=``,
  frame=no
}

\newcommand{\blue}{\textcolor{blue}}
\newcommand{\red}{\textcolor{red}}
\newcommand{\purple}{\textcolor{purple}}


\pgfrealjobname{survey}
\begin{document}
\begin{CJK}{UTF8}{gkai} 
  \beginpgfgraphicnamed{AlgProp}
  \begin{tikzpicture}
    The graphic
    \node at (0,0)[fill=blue!20,draw,starburst,drop shadow,text width=0.8cm]{
      算法特性
    };

    \node at (0,2) [right,fill=red!20,draw,rectangle callout,callout relative pointer={(-0.6,-0.5)},rounded corners=3mm,text width=3cm]{
      \blue{输入}\\
      有零个或多个输入
    };
    \node at (2,0.8) [right,fill=red!20,draw,ellipse callout,callout relative pointer={(-1.4,-0.3)},text width=3.6cm]{
      \blue{输出}\\
      至少有一个或多个输出
    };
    \node at (2,-1) [right,fill=red!20,draw,rectangle callout,callout relative pointer={(-1,0.1)},rounded corners=3mm,text width=5cm]{
      \blue{有穷性}\\
      在执行有限步后,会自动结束而不出现无限循环,并且每一步在可接受的时间内完成
    };

    \node at (1,-3.5) [right,fill=red!20,draw,ellipse callout,callout relative pointer={(-2.2,1.5)},text width=4cm]{
      \blue{确定性}\\
      每一步都有确定的含义,不出现二义性
    };
    \node at (-2,-3) [right,fill=red!20,draw,rectangle callout,callout relative pointer={(0.3,0.7)},rounded corners=3mm,text width=2.5cm]{
      \blue{可行性}\\
      每一步都必须可行,能通过执行有限次数完成
    };
  \end{tikzpicture}
  \endpgfgraphicnamed  
\end{CJK}

\end{document}


