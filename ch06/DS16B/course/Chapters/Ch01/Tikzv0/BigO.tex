\begin{figure}
  \centering
  \begin{tikzpicture}[node distance=3.2cm]
    \node [fill=blue!20,draw,starburst,drop shadow,text width=6cm,text=blue] (1) {
      如何分析一个算法的时间复杂度?即如何推导大O阶?
    };

    \pause 
    \node [below of=1,fill=red!20,draw,rectangle split,rectangle split parts=4,rounded corners=4mm,text width=8.5cm,text=blue]{
      1.~用常数1取代运行次数中的所有加法常数;
      \nodepart{two}
      2.~在修改后的运行次数函数中,只保留最高阶项;
      \nodepart{three}
      3.~如果最高阶项存在且不是1,则去除最高阶项的系数。
      \nodepart{four}
      得到的结果就是大O阶。
    };
  \end{tikzpicture}
  
\end{figure}
