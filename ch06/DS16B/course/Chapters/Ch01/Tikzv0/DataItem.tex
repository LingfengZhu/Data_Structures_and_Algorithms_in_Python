\documentclass{article}
\usepackage{CJK} 
\usepackage{graphics}
\usepackage{pgf}
\usepackage{tikz}
\usetikzlibrary{calc,shadows}
\usetikzlibrary{decorations.markings,scopes}
\usetikzlibrary{arrows,snakes,backgrounds,shapes}
\usetikzlibrary{decorations.pathmorphing}
\newcommand{\blue}{\textcolor{blue}}
\newcommand{\red}{\textcolor{red}}
\newcommand{\purple}{\textcolor{purple}}


\pgfrealjobname{survey}
\begin{document}
\begin{CJK}{UTF8}{gkai} 
  \beginpgfgraphicnamed{DataItem}
  \begin{tikzpicture}[scale=2,cap=round]

    % The graphic
    \node at (0,0)[fill=blue!20,draw,starburst,drop shadow,text width=4.5cm]
    {
      \blue{\large 数据项}: 一个元素可由若干个数据项组成.
    } ;

    \node at (-1,-1.6) [fill=blue!20,draw,ellipse callout, callout relative pointer={(0.7,0.7)},text width=5cm]
    { 
      如: 人可以有耳、鼻、眼、嘴、手、脚等这些数据项,也可以有姓名、年龄、性别、出生地址、联系电话等数据项.
    };

    \node at (2,-1.5) [text width=2cm,decorate,decoration=saw,fill=blue!20,draw,circle]
          {数据项是数据不可分割的最小部分.
          };

  \end{tikzpicture}
  \endpgfgraphicnamed  
\end{CJK}

\end{document}
