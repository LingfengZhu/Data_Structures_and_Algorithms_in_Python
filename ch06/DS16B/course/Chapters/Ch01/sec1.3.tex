\section{逻辑结构和物理结构}

\subsection{逻辑结构}
\begin{frame}
  \documentclass{article}
\usepackage{CJK} 
\usepackage{graphics}
\usepackage{pgf}
\usepackage{tikz}
\usetikzlibrary{calc,shadows}
\usetikzlibrary{decorations.markings,scopes}
\usetikzlibrary{arrows,snakes,backgrounds,shapes}
\usetikzlibrary{decorations.pathmorphing}
\newcommand{\blue}{\textcolor{blue}}
\newcommand{\red}{\textcolor{red}}
\newcommand{\purple}{\textcolor{purple}}


\pgfrealjobname{survey}
\begin{document}
\begin{CJK}{UTF8}{gkai} 
  \beginpgfgraphicnamed{LogicStruct}
  \begin{tikzpicture}[scale=2,cap=round]

    % The graphic
    \node at (0,0)[fill=blue!20,draw,starburst,drop shadow,text width=4.5cm]
    {
      \blue{\large 逻辑结构}: 指数据结构中数据元素之间的相互关系.
    } ;

    \node at (0,-1.6) [text width=2.3cm,decorate,decoration=saw,fill=blue!20,draw,circle]
          {
            \begin{itemize}
            \item 集合结构
            \item 线性结构
            \item 树形结构
            \item 图形结构
            \end{itemize}
          };

  \end{tikzpicture}
  \endpgfgraphicnamed  
\end{CJK}

\end{document}

\end{frame}

\begin{frame}
  \documentclass{article}
\usepackage{CJK} 
\usepackage{graphics}
\usepackage{pgf}
\usepackage{tikz}
\usetikzlibrary{calc,shadows}
\usetikzlibrary{decorations.markings,scopes}
\usetikzlibrary{arrows,snakes,backgrounds,shapes}
\usetikzlibrary{decorations.pathmorphing}
\newcommand{\blue}{\textcolor{blue}}
\newcommand{\red}{\textcolor{red}}
\newcommand{\purple}{\textcolor{purple}}


\pgfrealjobname{survey}
\begin{document}
\begin{CJK}{UTF8}{gkai} 
  \beginpgfgraphicnamed{SetStruct}
  
  \begin{tikzpicture}[scale=2,cap=round]

    % The graphic
    \node at (0,2)[fill=blue!20,draw,starburst,drop shadow,text width=6cm]
    {
      \blue{\large 集合结构}: 其中的元素除了同属于一个集合外,之间没有其他关系.
    } ;
    
    \draw (0,0)circle(1);
    \draw (0.00,0.00)node[]{1}circle(0.15);
    \draw (0.55,0.50)node[]{2}circle(0.15);
    \draw (-.65,-.10)node[]{3}circle(0.15);
    \draw (-.35,0.30)node[]{4}circle(0.15);
    \draw (0.80,0.10)node[]{5}circle(0.15);
    \draw (0.25,-.70)node[]{6}circle(0.15);
    \draw (-.15,0.70)node[]{7}circle(0.15);
    \draw (-.45,-.50)node[]{8}circle(0.15);
    \draw (0.55,-.20)node[]{9}circle(0.15);
  \end{tikzpicture}
  \endpgfgraphicnamed  
\end{CJK}

\end{document}

\end{frame}

\begin{frame}
  \begin{figure}  
  \centering
  \begin{tikzpicture}
    \tikzstyle{state}=[draw,circle]

        % The graphic
    \node at (-.5,2)[left,fill=blue!20,draw,starburst,drop shadow,text=red,text centered]
    (1){
      {线性结构}
    } ;
    \node at (.5,2)[right,fill=blue!20,draw,rectangle,rounded corners=3mm,text=blue,text width=7cm]
    (2){
      其中的数据元素之间是一对一的关系.
    }; 
    \pause    
    \node at (0,0) [state] (1) {1};
    \node[state] (2) [right of=1] {2};
    \node[state] (3) [right of=2] {3};
    \node[state] (4) [right of=3] {4};
    \node[state] (5) [right of=4] {5};
    \node[state] (6) [below of=5] {6};
    \node[state] (7) [below of=6] {7};
    \node[state] (8) [left  of=7] {8};
    \node[state] (9) [left  of=8] {9};
    \path 
    (1) edge (2)
    (2) edge (3)
    (3) edge (4)
    (4) edge (5)
    (5) edge (6)
    (6) edge (7)
    (7) edge (8)
    (8) edge (9);
  \end{tikzpicture}
\end{figure}

\end{frame}

\begin{frame}
  \begin{figure}  
  \centering
  \begin{tikzpicture}[level distance=1.5cm,
    level 1/.style={sibling distance=3cm},
    level 2/.style={sibling distance=1.5cm},
    ]
    \tikzstyle{state}=[draw,circle]
        % The graphic
    \node at (-.5,2)[left,fill=blue!20,draw,starburst,drop shadow,text=red,text centered]
    (1){
      {树形结构}
    } ;
    \node at (0.5,2)[right,fill=blue!20,draw,rectangle,rounded corners=3mm,text=blue,text width=6cm]
    (2){
      其中的数据元素之间是一对多的层次关系.
    };
    \pause
    
    \node[state]{A}
    child {node[state]{B}     
      child {node[state]{E}}
      child {node[state]{F}}
      child {node[state]{G}}
    }
    child {node[state]{C}
      child {node[state]{H}}
    }
    child {node[state]{D}     
      child {node[state]{I}}
      child {node[state]{J}}
    };
    
  \end{tikzpicture}
\end{figure}

\end{frame}

\begin{frame}
  \begin{figure}  
  \centering
  \begin{tikzpicture}[scale=1.5]
    \tikzstyle{state}=[draw,circle]

        % The graphic
    \node at (-.5,1.5)[left,fill=blue!20,draw,starburst,drop shadow,text=red,text centered]
    (1){
      {图形结构}
    } ;
    \node at (0.5,1.5)[right,fill=blue!20,draw,rectangle,rounded corners=3mm,text=blue,text width=6.5cm]
    (2){
      其中的数据元素之间是多对多的关系.
    };
    \pause
    \node at (0,0) [state] (1) {1};
    \node at (-1,-0.5) [state] (2)  {2};
    \node at (1.2,-0.5)[state] (3)  {3};
    \node at (-1.5,-1.5)[state] (4)  {4};
    \node at (0.2,-0.8)[state] (5)  {5};
    \node at (0.8,-2.0)[state] (7)  {7};
    \node at (1.8,-2.5)[state] (6)  {6};
    \node at (-0.5,-1.4)[state] (8)  {8};
    \node at (0.1,-2.8)[state] (9)  {9};
    \path 
    (1) edge (2) edge (3)
    (2) edge (4) edge (8) edge (5)
    (3) edge (5) edge (6)
    (4) edge (8) edge (9)
    (5) edge (7) edge (9)
    (6) edge (9)
    (7) edge (6) edge (9)
    (8) edge (9);
  \end{tikzpicture}
\end{figure}

\end{frame}

\begin{frame}
  在用示意图表示数据的逻辑结构时,请注意: \vspace{0.1in}
  
  \begin{itemize}
  \item 将每一个数据元素看做一个结点,用圆圈表示;\\[0.1in]
  \item 元素之间的逻辑关系用结点之间的连线表示,如果这个关系是有方向的,那么用带箭头的连线表示。
  \end{itemize}
  \pause

  \begin{figure}    
    \centering
    \begin{tikzpicture}
      \node [fill=red!20,starburst,drop shadow,draw,text width=7.5cm,text=blue]{
        逻辑结构是针对具体问题的,是为了解决某个问题。在对问题理解的基础上,选择一个合适的数据结构表示数据元素之间的逻辑关系。      
      };
    \end{tikzpicture}

  \end{figure}

\end{frame}


\subsection{物理结构}

\begin{frame}
  
  \documentclass{article}
\usepackage{CJK} 
\usepackage{graphics}
\usepackage{pgf}
\usepackage{tikz}
\usetikzlibrary{calc,shadows}
\usetikzlibrary{decorations.markings,scopes}
\usetikzlibrary{arrows,snakes,backgrounds,shapes}
\usetikzlibrary{decorations.pathmorphing}
\newcommand{\blue}{\textcolor{blue}}
\newcommand{\red}{\textcolor{red}}
\newcommand{\purple}{\textcolor{purple}}


\pgfrealjobname{survey}
\begin{document}
\begin{CJK}{UTF8}{gkai} 
  \beginpgfgraphicnamed{PhyStruct}
  \begin{tikzpicture}[scale=1.5]
    \tikzstyle{state}=[draw,circle]

        % The graphic
    \node at (0,0)[fill=blue!20,draw,starburst,drop shadow,text width=6.5cm]{
      \blue{\large 物理结构}: 指数据的逻辑结构在计算机中的存储方式.
    } ;

    \node at (-1,-2) [fill=blue!20,draw,ellipse callout, callout relative pointer={(0.,0.7)},text width=5.5cm]{ 
      数据的存储结构应正确反映数据元素之间的逻辑关系。如何存储数据元素之间的逻辑关系,是实现物理结构的重点和难点。
    };

    \node at (2.5,-1.3) [text width=1.5cm,decorate,decoration=saw,fill=blue!20,draw,circle,text=red]{
      顺序存储 链式存储
    };

    
  \end{tikzpicture}
  \endpgfgraphicnamed  
\end{CJK}

\end{document}



  
\end{frame}

\begin{frame}
  
  \documentclass{article}
\usepackage{CJK} 
\usepackage{graphics}
\usepackage{pgf}
\usepackage{tikz}
\usetikzlibrary{calc,shadows}
\usetikzlibrary{decorations.markings,scopes}
\usetikzlibrary{arrows,snakes,backgrounds,shapes}
\usetikzlibrary{decorations.pathmorphing}
\newcommand{\blue}{\textcolor{blue}}
\newcommand{\red}{\textcolor{red}}
\newcommand{\purple}{\textcolor{purple}}


\pgfrealjobname{survey}
\begin{document}
\begin{CJK}{UTF8}{gkai} 
  \beginpgfgraphicnamed{SeqStruct}
  \begin{tikzpicture}[scale=1.5]
    \tikzstyle{state}=[draw,circle]

        % The graphic
    \node at (0,2)[right,fill=blue!20,draw,starburst,drop shadow,text width=6.5cm]{
      \blue{\large 顺序存储结构} \\
      把数据元素存放在连续的存储单元里,其数据间的逻辑关系与物理关系一致。
    } ;

    %% \node at (-1,-2) [fill=blue!20,draw,ellipse callout, callout relative pointer={(0.,0.7)},text width=5.5cm]{ 
    %%   数据的存储结构应正确反映数据元素之间的逻辑关系。如何存储数据元素之间的逻辑关系,是实现物理结构的重点和难点。
    %% };

    %% \node at (2.5,-1.3) [text width=1.5cm,decorate,decoration=saw,fill=blue!20,draw,circle,text=red]{
    %%   顺序存储 链式存储
    %% };
    \def\xx{0.8}
    \draw[very thick] (0,0)--(9*\xx,0);
    \foreach \i in {0,...,9} 
    \draw[very thick] (\i*\xx,0)--(\i*\xx,1*\xx);      
    \foreach \i in {1,...,9} 
    \draw[thick] (\i*\xx-0.5*\xx,0.5*\xx)node[]{\i}circle(0.5*\xx);

    
  \end{tikzpicture}
  \endpgfgraphicnamed  
\end{CJK}

\end{document}



  
\end{frame}

\begin{frame}
  
  \documentclass{article}
\usepackage{CJK} 
\usepackage{graphics}
\usepackage{pgf}
\usepackage{tikz}
\usetikzlibrary{calc,shadows}
\usetikzlibrary{decorations.markings,scopes}
\usetikzlibrary{arrows,snakes,backgrounds,shapes}
\usetikzlibrary{decorations.pathmorphing}
\newcommand{\blue}{\textcolor{blue}}
\newcommand{\red}{\textcolor{red}}
\newcommand{\purple}{\textcolor{purple}}


\pgfrealjobname{survey}
\begin{document}
\begin{CJK}{UTF8}{gkai} 
  \beginpgfgraphicnamed{LinkStruct}
  \begin{tikzpicture}[->,>=stealth,scale=1.5]
    \tikzstyle{state}=[draw,circle]

        % The graphic
    \node at (0,2)[right,fill=blue!20,draw,starburst,drop shadow,text width=6.5cm]{
      \blue{\large 链式存储结构} \\
      把数据元素存放在任意的存储单元里,这组存储单元可以连续,也可以不连续。
    } ;

    \node at (5,-1) [fill=blue!20,draw,ellipse callout, callout relative pointer={(0.,0.7)},text width=4cm]{ 
      数据元素的存储关系并不能反映其逻辑关系,需要用一个指针存放数据元素的地址,这样就可以通过地址找到相关联数据元素的位置。
    };

    
    \node at (0,0)    [state] (1) {1};
    \node at (2.3,-0.3) [state] (2)  {2};
    \node at (2.7,0.1) [state] (9)  {9};
    \node at (1.6,-0.6) [state] (3)  {3};
    \node at (1.0,-1.2) [state] (5)  {5};
    \node at (0.0,-1.8) [state] (8)  {8};
    \node at (1.3,-2.5) [state] (6)  {6};
    \node at (2.4,-2.3) [state] (4)  {4};
    \node at (2.6,-1.0) [state] (7)  {7};
    
    \path (1) edge [bend left] (2)
    (2) edge [bend right] (3)
    (3) edge [bend left] (4)
    (4) edge [bend left] (5)
    (5) edge [bend right] (6)
    (6) edge [bend right] (7)
    (7) edge [bend left] (8)
    (8) edge [bend left] (9)
    ;
    
  \end{tikzpicture}
  \endpgfgraphicnamed  
\end{CJK}

\end{document}



  
\end{frame}

\begin{frame}
  
  \documentclass{article}
\usepackage{CJK} 
\usepackage{graphics}
\usepackage{pgf}
\usepackage{tikz}
\usetikzlibrary{calc,shadows}
\usetikzlibrary{decorations.markings,scopes}
\usetikzlibrary{arrows,snakes,backgrounds,shapes}
\usetikzlibrary{decorations.pathmorphing}
\newcommand{\blue}{\textcolor{blue}}
\newcommand{\red}{\textcolor{red}}
\newcommand{\purple}{\textcolor{purple}}


\pgfrealjobname{survey}
\begin{document}
\begin{CJK}{UTF8}{gkai} 
  \beginpgfgraphicnamed{LogicPhySummary}
  \begin{tikzpicture}
  \node[copy shadow,fill=blue!20,draw=blue,thick,text width=7cm] at (3.5,0) {
    逻辑结构是面向问题的,而物理结构是面向计算机的,我们的目标是将数据及其逻辑关系存储到计算机的内存中。
  };
\end{tikzpicture}
  \endpgfgraphicnamed  
\end{CJK}

\end{document}



  
\end{frame}
