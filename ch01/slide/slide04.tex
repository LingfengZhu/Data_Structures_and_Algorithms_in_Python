\section{什么是编程}

\begin{frame}\ft{\secname}
编程是将算法转换为程序语言的过程,以便能被计算机所执行。编程语言有很多,计算机的种类也有不同,我们首先要有解决方案,因没有算法就没有程序。
% \end{frame}

% \begin{frame}\ft{\secname}
计算机科学不研究编程,但编程却是计算机科学家的重要能力。编程通常是解决方案的表达方式。
\end{frame}

\begin{frame}\ft{\secname}
算法描述了依据问题实例数据所产生的解决方案和产生预期结果所需的一套步骤。编程语言必须提供一种表示方法来表示过程和数据。为此,它提供了\red{控制结构}和\red{数据类型}。
\end{frame}

\begin{frame}\ft{\secname}
\red{控制结构}允许以方便而明确的方式表示算法步骤。至少,算法需要执行顺序处理、决策选择和重复控制迭代。只要语言提供这些基本语句,它就可以表达算法。
\end{frame}

\begin{frame}\ft{\secname}
计算机中的所有数据项都由一串一串的二进制数表示。为了让这些二进制串有意义,就需要有\red{数据类型}。数据类型为二进制数据提供解释,以便我们能够根据实际问题来思考数据。这些底层的内置数据类型(有时称为原始数据类型)为算法开发提供了基础。
\end{frame}

\begin{frame}\ft{\secname}
例如,大多数编程语言为整数提供数据类型。内存中的二进制数据可以解释为整数,并且能给予一个我们通常与整数(例如 23,654 和 -19)相关联的含义。此外,数据类型还提供数据项参与的操作的描述。对于整数,诸如加法、减法和乘法的操作是常见的。我们期望数值类型的数据可以参与这些算术运算。
\end{frame}

\begin{frame}\ft{\secname}
通常我们遇到的困难是问题及其解决方案非常复杂。由语言提供的简单的结构和数据类型,虽然可以表示复杂的解决方案,但在实际中却不好用。我们需要一些方法控制这种复杂性,以助于形成更好的解决方案。
\end{frame}