\section{什么是编程}
Programming is the process of taking an algorithm and encoding it into a notation, a programming language, so that it can be executed by a computer. Although many programming languages and many different types of computers exist, the important first step is the need to have the solution. Without an algorithm there can be no program.

编程是将算法转换为编程语言的过程,以便被计算机执行。然而,编程语言有很多,计算的种类也不同,首先我们要有解决方案,没有算法就没有程序。

Computer science is not the study of programming. Programming, however, is an important part of what a computer scientist does. Programming is often the way that we create a representation for our solutions. Therefore, this language representation and the process of creating it becomes a fundamental part of the discipline.

计算机科学不研究编程,但编程却是计算机科学家的重要能力。编程通常是表达解决方案的方式。因此,这种语言表现形式和创造它的过程成为该学科的基本部分。

Algorithms describe the solution to a problem in terms of the data needed to represent the problem instance and the set of steps necessary to produce the intended result. Programming languages must provide a notational way to represent both the process and the data. To this end, languages provide control constructs and data types.

算法描述了依据问题实例数据所产生的解决方案和产生预期结果所需的一套步骤。编程语言必须提供一种表示方法来表示过程和数据。为此,它提供了控制结构和数据类型。

Control constructs allow algorithmic steps to be represented in a convenient yet unambiguous way. At a minimum, algorithms require constructs that perform sequential processing, selection for decision-making, and iteration for repetitive control. As long as the language provides these basic statements, it can be used for algorithm representation.

控制结构允许以方便而明确的方式表示算法步骤。至少,算法需要执行顺序处理、决策选择和重复控制迭代。只要语言提供这些基本语句,它就可以表达算法。

All data items in the computer are represented as strings of binary digits. In order to give these strings meaning, we need to have data types. Data types provide an interpretation for this binary data so that we can think about the data in terms that make sense with respect to the problem being solved. These low-level, built-in data types (sometimes called the primitive data types) provide the building blocks for algorithm development.

计算机中的所有数据项都由一串一串的二进制数表示。为了让这些二进制串有意义,就需要有数据类型。数据类型为二进制数据提供解释,以便我们能够根据实际问题来思考数据。这些底层的内置数据类型(有时称为原始数据类型)为算法开发提供了基础。

For example, most programming languages provide a data type for integers. Strings of binary digits in the computer’s memory can be interpreted as integers and given the typical meanings that we commonly associate with integers (e.g. 23, 654, and -19). In addition, a data type also provides a description of the operations that the data items can participate in. With integers, operations such as addition, subtraction, and multiplication are common. We have come to expect that numeric types of data can participate in these arithmetic operations.

例如,大多数编程语言为整数提供数据类型。内存中的二进制数据可以解释为整数,并且能给予一个我们通常与整数(例如 23,654 和 -19)相关联的含义。此外,数据类型还提供数据项参与的操作的描述。对于整数,诸如加法、减法和乘法的操作是常见的。我们期望数值类型的数据可以参与这些算术运算。

The difficulty that often arises for us is the fact that problems and their solutions are very complex. These simple, language-provided constructs and data types, although certainly sufficient to represent complex solutions, are typically at a disadvantage as we work through the problem-solving process. We need ways to control this complexity and assist with the creation of solutions.

通常我们遇到的困难是问题及其解决方案非常复杂。由语言提供的简单的结构和数据类型,虽然可以表示复杂的解决方案,但在实际中却不好用。我们需要一些方法控制这种复杂性,以助于形成更好的解决方案。