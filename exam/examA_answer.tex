%This is a LaTeX template for homework assignments
\documentclass[8pt]{article}
\usepackage[top=1in, bottom=1in, left=1.1in, right=1.1in]{geometry}
\usepackage[utf8]{inputenc}
\usepackage{amsmath}
\usepackage{CJK}
\usepackage{enumerate}
\usepackage{ifthen}
\usepackage{listings}
\lstset{
  language=python,
  keywordstyle=\color{blue!70},
  frame=single,
  basicstyle=\ttfamily,
  commentstyle=\color{red},
  breakindent=0pt,
  rulesepcolor=\color{red!20!green!20!blue!20},
  rulecolor=\color{black},
  tabsize=4,
  numbersep=5pt,
  showstringspaces=false,
  breaklines=true,
  backgroundcolor=\color{red!10},
  showspaces=false,
  showtabs=false,
  extendedchars=false,
  escapeinside=``,
  frame=no,
}

\usepackage{fourier}
\usepackage{pgf}
\usepackage{tikz}
\usetikzlibrary{calc}
\usetikzlibrary{arrows,snakes,backgrounds,shapes,shadows}
\usetikzlibrary{matrix,fit,positioning,decorations.pathmorphing}



\newcommand{\tf}{\ttfamily}


\newlength{\la}
\newlength{\lb}
\newlength{\lc}
\newlength{\ld}
\newlength{\lhalf}
\newlength{\lquarter}
\newlength{\lmax}
\newcommand{\xx}[4]{\\[.5pt]%
\settowidth{\la}{A.~#1~~}
\settowidth{\lb}{B.~#2~~}
\settowidth{\lc}{C.~#3~~}
\settowidth{\ld}{D.~#4~~}
%%
\ifthenelse{\lengthtest{\la>\lb}}
{\setlength{\lmax}{\la}}
{\setlength{\lmax}{\lb}}
\ifthenelse{\lengthtest{\lmax<\lc}}
{\setlength{\lmax}{\lc}}
{}
\ifthenelse{\lengthtest{\lmax<\ld}}
{\setlength{\lmax}{\ld}}
{}
%%
\setlength{\lhalf}{0.5\linewidth}
\setlength{\lquarter}{0.25\linewidth}
%%
\ifthenelse{\lengthtest{\lmax>\lhalf}}
{\noindent{}A.~#1 \\ B.~#2 \\ C.~#3 \\ D.~#4 }
{
\ifthenelse{\lengthtest{\lmax>\lquarter}}
{\noindent
\makebox[\lhalf][l]{A.~#1~~}%
\makebox[\lhalf][l]{B.~#2~~}\\
\makebox[\lhalf][l]{C.~#3~~}%
\makebox[\lhalf][l]{D.~#4~~}
}%
{\noindent
\makebox[\lquarter][l]{A.~#1~~}%
\makebox[\lquarter][l]{B.~#2~~}%
\makebox[\lquarter][l]{C.~#3~~}%
\makebox[\lquarter][l]{D.~#4~~}
}
}
}



\begin{document}
\begin{CJK}{UTF8}{gkai}
\begin{center}
{\Large \bf  武汉大学数学与统计学院2017-2018学年第一学期期末考试\\[0.1in]
  数据结构与算法(A卷)} \vspace{0.1in}

姓名: \line(1,0){100} ~~~~~~ 学号: \line(1,0){100}

\end{center}


\begin{enumerate} %\line(1,0){20}
\item Python相关
  \begin{enumerate}
  \item 给定长(\lstinline{length})和宽(\lstinline{width}),编写长方形(\lstinline{Rectangle})的类,并求其面积和周长。
  \item 仔细阅读以下程序,写出运行结果:
    
  \end{enumerate}
\item 编写程序,实现栈的抽象数据类型。
\item 二叉树
  \begin{enumerate}
  \item 编写代码,实现队列类:
    \begin{lstlisting}
class Queue:
    def __init__(self):
        self.items = []
    def isEmpty(self):
        _______________
    def enqueue(self, item):
        _______________
    def dequeue(self):
        _______________    
    def size(self):
        _______________
    \end{lstlisting}
  \item 编写代码,利用结点-引用方式来实现二叉树类:
    \begin{lstlisting}
class BinaryTree(object):
    def __init__(self, rootObj):
        self.data = rootObj
        self.lchild = None
        self.rchild = None
    \end{lstlisting}
  \item 编写代码,实现二叉树的中序遍历:
    \begin{lstlisting}
def inorder(root):
    ...
    \end{lstlisting}
  \item 编写代码,利用队列类实现二叉树的层序遍历:
    \begin{lstlisting}
def levelorder(root):
    if root == None:
       return 
    q = Queue()
    node = root
    q.enqueue(node)
    while not q.isEmpty():
        node = q.dequeue()
        print(node.data)
        if node.lchild != None
            q.enquque(node.lchild)
        if node.rchild != None:
            q.enqueue(node.rchilde)
      \end{lstlisting}
  \end{enumerate}
\end{enumerate}



\end{CJK}
\end{document}
