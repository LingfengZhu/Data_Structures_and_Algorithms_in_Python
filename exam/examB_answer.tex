%This is a LaTeX template for homework assignments
\documentclass[8pt]{article}
\usepackage[top=1in, bottom=1in, left=1.1in, right=1.1in]{geometry}
\usepackage[utf8]{inputenc}
\usepackage{amsmath}
\usepackage{CJK}
\usepackage{enumerate}
\usepackage{ifthen}
\usepackage{listings}
\lstset{
  language=c,
  keywordstyle=\color{blue!70},
  frame=single,
  basicstyle=\ttfamily\footnotesize,
  commentstyle=\color{red},
  breakindent=0pt,
  rulesepcolor=\color{red!20!green!20!blue!20},
  rulecolor=\color{black},
  tabsize=4,
  numbersep=5pt,
  showstringspaces=false,
  breaklines=true,
  backgroundcolor=\color{red!10},
  showspaces=false,
  showtabs=false,
  extendedchars=false,
  escapeinside=``,
  frame=no,
}

\usepackage{fourier}
\usepackage{pgf}
\usepackage{tikz}
\usetikzlibrary{calc}
\usetikzlibrary{arrows,snakes,backgrounds,shapes,shadows}
\usetikzlibrary{matrix,fit,positioning,decorations.pathmorphing}



\newcommand{\tf}{\ttfamily}


\newlength{\la}
\newlength{\lb}
\newlength{\lc}
\newlength{\ld}
\newlength{\lhalf}
\newlength{\lquarter}
\newlength{\lmax}
\newcommand{\xx}[4]{\\[.5pt]%
\settowidth{\la}{A.~#1~~}
\settowidth{\lb}{B.~#2~~}
\settowidth{\lc}{C.~#3~~}
\settowidth{\ld}{D.~#4~~}
%%
\ifthenelse{\lengthtest{\la>\lb}}
{\setlength{\lmax}{\la}}
{\setlength{\lmax}{\lb}}
\ifthenelse{\lengthtest{\lmax<\lc}}
{\setlength{\lmax}{\lc}}
{}
\ifthenelse{\lengthtest{\lmax<\ld}}
{\setlength{\lmax}{\ld}}
{}
%%
\setlength{\lhalf}{0.5\linewidth}
\setlength{\lquarter}{0.25\linewidth}
%%
\ifthenelse{\lengthtest{\lmax>\lhalf}}
{\noindent{}A.~#1 \\ B.~#2 \\ C.~#3 \\ D.~#4 }
{
\ifthenelse{\lengthtest{\lmax>\lquarter}}
{\noindent
\makebox[\lhalf][l]{A.~#1~~}%
\makebox[\lhalf][l]{B.~#2~~}\\
\makebox[\lhalf][l]{C.~#3~~}%
\makebox[\lhalf][l]{D.~#4~~}
}%
{\noindent
\makebox[\lquarter][l]{A.~#1~~}%
\makebox[\lquarter][l]{B.~#2~~}%
\makebox[\lquarter][l]{C.~#3~~}%
\makebox[\lquarter][l]{D.~#4~~}
}
}
}



\begin{document}
\begin{CJK}{UTF8}{gkai}
\begin{center}
{\Large \bf  武汉大学数学与统计学院2015-2016学年第二学期期末考试\\[0.1in]C语言程序设计(B卷)\\[0.1in]标准答案} \vspace{0.1in}

%姓名: \line(1,0){100} ~~~~~~ 学号: \line(1,0){100}

\end{center}

%% \section*{2015-2016学年武汉大学期末考试\\C语言程序设计}

%\subsection*{说明} %Enter instruction text here
%Answer {\bf all} the questions and return to the teacher by the end of the week. Section A questions shouldn't cause you any trouble. Section B is designed to be more challenging.

\subsection*{一、单选题(每题2分,共20分)}

\begin{table}[htbp]
\centering
\begin{tabular}{|c|c|c|c|c|c|c|c|c|c|}\hline
1&2&3&4&5&6&7&8&9&10\\\hline
D&D&A&D&C&C&C&A&D&D \\\hline
\end{tabular}
\end{table}


%\begin{enumerate}
%\item D
%%下面程序
%%\begin{lstlisting}
%%int main(void)
%%{
%%  int k = 11;
%%  printf("k=%d,k=%o,k=%x\n",k,k,k);
%%}
%%\end{lstlisting}
%%的输出是\line(1,0){20}。
%%\xx
%%{k=11,k=12,k=11}
%%{k=11,k=13,k=13}
%%{k=11,k=013,k=0xb}
%%{k=11,k=13,k=b}
%
%
%\item D
%%设int a = 12, 则执行完语句a+=a-=a*a;后,a的值为\line(1,0){20}。
%%\xx
%%{552}
%%{264} 
%%{144}
%%{-264}
%
%\item A
%%为表达关系$x\ge y \ge z$,应使用C表达式\line(1,0){20}
%%\xx{(x >= y) \&\& (y >= z)}
%%{(x >= y) AND (y >= z) 300)}
%%{(x >= y >= z))}
%%{(x >= y) \& (y >= z) 300)}
%
%\item D
%%设d为double型变量,则逗号表达式\begin{lstlisting}
%%d=1, d+5, d++    
%%\end{lstlisting}的值是\line(1,0){20}
%%\xx{1}
%%{6.0}
%%{2.0}
%%{1.0}
%
%\item C
%%以下程序段
%%\begin{lstlisting}
%%x = -1;
%%do {
%%  x = x*x;
%%} while (!x);
%%\end{lstlisting}
%%描述正确的是\line(1,0){20}。
%%\xx{是死循环}
%%{循环执行两次}
%%{循环执行一次}
%%{有语法错误}
%
%\item C
%%以下描述正确的是\line(1,0){20}。
%%\xx
%%{goto语句只能用于退出多层循环}
%%{switch语句中不能出现continue语句}
%%{只能用continue语句来终止本次循环}
%%{在循环中break语句不能独立出现}
%
%\item C
%%设x、y、z、t均为int型变量,则执行以下语句后,
%%\begin{lstlisting}
%%x = y = z = 1;
%%t = ++x || ++y && ++z;
%%\end{lstlisting}    
%%t的值为\line(1,0){20}。
%%\xx{不定值}
%%{4}
%%{1}
%%{0}
%
%\item A
%%C语言中,凡未指定存储类别的局部变量的隐含存储类别是\line(1,0){20}。
%%\xx
%%{auto}
%%{static}
%%{extern}
%%{register}  
%
%
%\item D
%%若有说明int  a[3][4]={0};则下面正确的叙述是\line(1,0){20}。
%%\xx
%%{只有元素a[0][0]可得到初值0}
%%{此说明语句不正确}
%%{数组a中各元素都可得到初值,但其值不一定为0}
%%{数组a中每个元素均可得到初值0 }
%
%\item D
%%定义如下结构体数组
%%\begin{lstlisting}
%%struct {
%%  int num;
%%  char name[10];
%%} x[3] = {1,"China",2,"USA",3,"England"};
%%\end{lstlisting} 
%%语句
%%\begin{lstlisting}
%%printf("\n%d,%s",x[1].num,x[2].name);
%%\end{lstlisting} 
%%的输出结果是\line(1,0){20}D。
%%\xx{2,USA}
%%{3,England}
%%{1,China}
%%{2,England}
%\end{enumerate}

\subsection*{二、程序阅读题(30分)}
\begin{enumerate}
\item (6分)
%下列程序将打印出什么?
%\begin{lstlisting}[showstringspaces=true]
%#include <stdio.h>
%#define TEN 10
%int main(void)
%{
% int n = 0;
% while (n++ < TEN) 
%   printf("%5d", n);
% printf("\n");
% return 0;
%}
%\end{lstlisting}
\begin{lstlisting}[showspaces=true]
    1    2    3    4    5    6    7    8    9   10
\end{lstlisting}

\item (6分)
%下列程序将打印出什么?
%\begin{lstlisting}[showstringspaces=true]
%#include <stdio.h>
%int main(void)
%{
% int k;
% for (k = 1, printf("%d: Hi!\n", k); 
%      printf("k = %d\n", k), k*k < 26;
%      k += 2, printf("Now k is %d, ", k))
%     printf("k is %d in the loop\n", k); 
% return 0;
%}
%\end{lstlisting}
\begin{lstlisting} 
1: Hi!
k = 1
k is 1 in the loop
Now k is 3, k = 3
k is 3 in the loop
Now k is 5, k = 5
k is 5 in the loop
Now k is 7, k = 7
\end{lstlisting}

\item (8分)
%用适当的方法声明下面每个变量:
%\begin{itemize}
%\item[a.] digits: 一个包含10个int值的数组
%\item[b.] rates: 一个包含6个float值的数组
%\item[c.] mat: 一个包含3个元素的数组,其中每个元素是一个包含5个整数的数组
%\item[d.] pstr: 一个指向数组的指针,其中数组由20个char值构成
%\end{itemize}

\begin{lstlisting} 
a. int digits[10];
b. float rates[6];
c. int mat[3][5];
d. char (*pstr)[20];
\end{lstlisting}

\item (5分)
%文件开始处做了如下声明:
%\begin{lstlisting} 
%static int plink;
%int value_ct(const int arr[], int value, int n);
%\end{lstlisting}    
%\begin{itemize}
%\item[a.] 这些声明表明了程序员的什么意图?
%\item[b.] 用const int value和const int n代替int value和int n会增强对调用程序中值的保护吗?
%\end{itemize}
\begin{itemize}
\item[a.]
第一条语句表明整型变量plink仅在当前文件可见,第二条声明语句表明在函数value\_ct中,数组arr是被保护的。
\item[b.] 
可以增强调用程序中值的保护。
\end{itemize}
\item (5分)
%你会用什么转换说明来打印下列各项内容?
%\begin{itemize}
%\item[a.] 一个字段宽度为6、最少有4位数字的十进制整数
%\item[b.] 一个字段宽度在参数列表中给定的八进制整数
%\item[c.] 一个字段宽度为2的字符
%\item[d.] 一个形如+3.13、字段宽度等于数字中字符个数的浮点数
%\item[e.] 一个字符串的前5个字符,字段宽度为7、左对齐
%\end{itemize} 

\begin{itemize}
\item[a.] \%6.4d 
\item[b.] \%*o 
\item[c.] \%2c 
\item[d.] \%+0.2f 
\item[e.] \%-7.5s
\end{itemize}   
\end{enumerate}


\subsection*{三、问答题(25分)}

\begin{enumerate}
\item (6分)
%简述while循环、do while循环与for循环的区别。
\begin{itemize}
\item
相同点:都是进行循环判断的
\item
不同点:
\begin{itemize}
\item
do-while是先执行后判断,因此do-while至少要执行一次循环体。
\item
而while是先判断后执行,如果条件不满足,则一次循环体语句也不执行。
\item 
对于for循环
\begin{lstlisting}
for(`表达式1`; `表达式2`; `表达式3`)
\end{lstlisting}
第一步,计算表达式1的值。第二步,计算表达式2的值。若值为真(非0)则执行循环体一次,否则跳出循环。第三步,计算表达式3的值,转回第二步重复执行。
\end{itemize}
\end{itemize}
\item (6分)
%下面这个字符串的声明错在哪里?
%\begin{lstlisting}
%int main(void) 
%{
%  char name[] = {'F', 'i', 'n', 'e',};
%  ...
%}
%\end{lstlisting}
初始化列表中缺少空字符,这导致name仅仅为一个字符数组,而非字符串。
\item (4分)
%看看下面这段程序有什么错误:
%\begin{lstlisting}
%void swap(int * p1, int * p2)
%{
%  int * p;
%  *p = 4;
%}
%\end{lstlisting}
声明指针p时未作初始化,这使得p可能指向内存的任意位置,当把4赋值给p所指向的存储位置时,可能导致程序崩溃。
\item (9分)
分别说明以下声明语句的含义
\begin{lstlisting}
(1) int (*uuf[3])[4];
(2) int (*oof)[3][4]; 
(3) char (* f1[3])();
(4) char * f2(int a, int b);
(5) typedef int arr5[5];
    typedef arr5 * p_arr5;
    p_arr5 p2;
\end{lstlisting}
\begin{itemize}
\item[(1)] uuf是一个长度为3的数组,各元素均为指向长度为4的数组的指针;
\item[(2)] oof是一个指向$3\times4$二维数组的指针;
\item[(3)] f1是一个长度为3的数组,各元素均为指向返回值为char、不接受任何参数的函数的指针;
\item[(4)] f2是返回值为指向int的指针、接受两个int参数的函数;
\item[(5)] p2为一个指向长度为5的int数组的指针。
\end{itemize}
\end{enumerate}


\subsection*{四、编程题(25分)}

\begin{enumerate}
\item (5分)
%编制程序,对于给定的一个百分制成绩,输出相应的5分制乘积。设:90分以上为'A',80-89分为'B',70-79分为'C',60-69分为'D',60分以下为'E'。(用switch语句实现)
\lstinputlisting[]{examB_code/ex4_1.c}

\item (5分)
%从键盘输入两个整数,输出其最大公约数和最小公倍数。
\lstinputlisting[]{examB_code/ex4_2.c}

\item (6分)
%输入任意5个数放在数组中,假如5个数为1、2、8、3、10,请打印以下方阵
%$$
%\left(
%\begin{array}{rrrrr}
%1&2&8&3&10\\
%2&8&3&10&1\\
%8&3&10&1&2\\
%3&10&1&2&8\\
%10&1&2&8&3\\
%\end{array}
%\right)
%$$
\lstinputlisting[]{examB_code/ex4_3.c}


\item (9分)
%求一个一维数组的最大值,最小值和平均值(主函数调用并输出)。
\lstinputlisting[]{examB_code/ex4_4.c}

\end{enumerate}


\end{CJK}
\end{document}
