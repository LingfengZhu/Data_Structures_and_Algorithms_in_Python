\section{一个乱序字符串检查的例子}

显示不同量级的算法的一个很好的例子是字符串的乱序检查。乱序字符串是指一个字符串只是另一个字符串的重新排列。例如,'heart' 和 'earth' 就是乱序字符串。'python' 和 'typhon' 也是。为了简单起见,我们假设所讨论的两个字符串具有相等的长度,并且他们由 26 个小写字母集合组成。我们的目标是写一个布尔函数,它将两个字符串做参数并返回它们是不是乱序。


\subsection{解法1:检查}

第一种方法是检查第一个字符串是不是出现在第二个字符串中。如果可以检验到每一个字符,那这两个字符串一定是乱序。可以通过用 None 替换字符来了解一个字符是否完成检查。但是,由于 Python 字符串是不可变的,所以第一步是将第二个字符串转换为列表。检查第一个字符串中的每个字符是否存在于第二个列表中,如果存在,替换成 None。 

\lstinputlisting{code/anagramSolution1.py}  


为了分析这个算法,我们注意到 $s_1$ 的每个字符都会在 $s_2$ 中进行最多 $n$ 个字符的迭代。$s_2$ 列表中的 $n$ 个位置将被访问一次来匹配来自 $s_1$ 的字符。访问次数可以写成 $1$ 到 $n$ 整数的和,可以写成 
$$
\sum_{i=1}^n = \frac{n(n+1)}{2}=\frac{n^2}{2}+\frac{n}{2}.
$$
当 $n$ 变大,$n^2$ 这项占据主导,$1/2$ 可以忽略。所以这个算法复杂度为 $O(n^2 )$。


\subsection{解法2:排序和比较}

另一个解决方案是利用这么一个事实:即使 $s_1,s_2$ 不同,它们都是由完全相同的字符组成的。所以,我们按照字母顺序从 $a$ 到 $z$ 排列每个字符串,如果两个字符串相同,那这两个字符串就是乱序字符串。 

\lstinputlisting{code/anagramSolution2.py}  

首先你可能认为这个算法是 $O(n)$,因为只有一个简单的迭代来比较排序后的 $n$ 个字符。但是,调用 Python 排序不是没有成本。正如我们将在后面的章节中看到的,排序通常是 $O(n^2)$ 或 $O(n\log n)$。所以排序操作比迭代花费更多。最后该算法跟排序过程有同样的量级。


\subsection{解法3: 穷举法}

解决这类问题的强力方法是穷举所有可能性。对于乱序检测,我们可以生成 $s_1$ 的所有乱序字符串列表,然后查看是不是有 $s_2$。这种方法有一点困难。当 $s_1$ 生成所有可能的字符串时,第一个位置有 $n$ 种可能,第二个位置有 $n-1$ 种,第三个位置有 $n-2$ 种,等等。总数为 $n\cdot(n−1)\cdot(n−2)\cdots 3\cdot 2\cdot 1=n!$。虽然一些字符串可能是重复的,程序也不可能提前知道这样,所以他仍然会生成 $n!$ 个字符串。
事实证明,$n!$ 比 $n^2$ 增长还快,事实上,如果 $s_1$ 有 $20$ 个字符长,则将有 $20! = 2,432,902,008,176,640,000$ 个字符串产生。如果我们每秒处理一种可能字符串,那么需要 $77,146,816,596$ 年才能过完整个列表。所以这不是很好的解决方案。


\subsection{解法4: 计数和比较}

我们最终的解决方法是利用两个乱序字符串具有相同数目的 $a, b, c$ 等字符的事实。我们首先计算的是每个字母出现的次数。由于有 $26$ 个可能的字符,我们就用 一个长度为 $26$ 的列表,每个可能的字符占一个位置。每次看到一个特定的字符,就增加该位置的计数器。最后如果两个列表的计数器一样,则字符串为乱序字符串。 

\lstinputlisting{code/anagramSolution4.py}  

同样,这个方案有多个迭代,但是和第一个解法不一样,它不是嵌套的。两个迭代都是 n, 第三个迭代,比较两个计数列表,需要 $26$ 步,因为有 $26$ 个字母。一共 $T(n)=2n+26T(n)=2n+26$,即 $O(n)$,我们找到了一个线性量级的算法解决这个问题。


在结束这个例子之前,我们来讨论下空间花费,虽然最后一个方案在线性时间执行,但它需要额外的存储来保存两个字符计数列表。换句话说,该算法牺牲了空间以获得时间。
很多情况下,你需要在空间和时间之间做出权衡。这种情况下,额外空间不重要,但是如果有数百万个字符,就需要关注下。作为一个计算机科学家,当给定一个特定的算法,将由你决定如何使用计算资源。
