\documentclass{article}
\usepackage{CJK} 
\usepackage{graphics}
\usepackage{pgf}
\usepackage{tikz}
\usetikzlibrary{calc,shadows}
\usetikzlibrary{decorations.markings,scopes}
\usetikzlibrary{arrows,snakes,backgrounds,shapes}
\usetikzlibrary{decorations.pathmorphing}
\newcommand{\blue}{\textcolor{blue}}
\newcommand{\red}{\textcolor{red}}
\newcommand{\purple}{\textcolor{purple}}


\pgfrealjobname{survey}
\begin{document}
\begin{CJK}{UTF8}{gkai} 
  \beginpgfgraphicnamed{ADTExample}
  \begin{tikzpicture}
    % The graphic
    \node at (0,0)[fill=blue!20,draw,starburst,drop shadow]{
      例
    };

    \node at (0,-2) [right,fill=blue!20,draw,ellipse,text width=7cm]{
      编写计算机绘图软件时,经常会用到坐标,总会出现成对的$x$和$y$,在3D系统中还有$z$出现。我们不妨定义一个叫point的抽象数据类型,
      它有$x, y, z$三个变量。这样我们可以方便地操作一个point数据变量就能知道这一点的坐标。
    };

  \end{tikzpicture}
  \endpgfgraphicnamed  
\end{CJK}

\end{document}


