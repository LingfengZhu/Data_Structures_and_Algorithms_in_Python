\documentclass{article}
\usepackage{CJK} 
\usepackage{graphics}
\usepackage{pgf}
\usepackage{tikz}
\usetikzlibrary{calc,shadows}
\usetikzlibrary{decorations.markings,scopes}
\usetikzlibrary{arrows,snakes,backgrounds,shapes}
\usetikzlibrary{decorations.pathmorphing}
\newcommand{\blue}{\textcolor{blue}}
\newcommand{\red}{\textcolor{red}}
\newcommand{\purple}{\textcolor{purple}}


\pgfrealjobname{survey}
\begin{document}
\begin{CJK}{UTF8}{gkai} 
  \beginpgfgraphicnamed{Abstract}
  \begin{tikzpicture}
    % The graphic
    \node at (0,0)[fill=blue!20,draw,starburst,drop shadow,text width=6cm]{
      \blue{\large 抽象}\\
      抽取出事物具有的普遍性的本质。
    };

    \node at (0,-3) [fill=blue!20,draw,ellipse,text width=7cm]{
      提炼出问题的特征,忽略非本质的细节,对具体事物做一个概括。\vspace{0.1in}

      抽象是一种思考问题的方式,它隐藏了繁杂的细节,只保留实现目标所必须的信息。
    };

  \end{tikzpicture}
  \endpgfgraphicnamed  
\end{CJK}

\end{document}


