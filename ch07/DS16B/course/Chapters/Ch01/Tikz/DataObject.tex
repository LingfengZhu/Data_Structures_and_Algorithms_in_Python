\begin{figure}
  \centering
  \begin{tikzpicture}

    % The graphic
    \node at (1.0,0)[left,fill=blue!20,draw,starburst,drop shadow,text=blue]
    (1){
      {\large 数据对象} 
    };
    \node at (2,0)[right,fill=blue!20,draw,rectangle,rounded corners=3mm,text width=6cm]
    (2){
      性质相同的数据元素的集合,它是数据的子集。
    };
    \pause 
    \node at (-2,-3) [right,fill=red!20,draw,ellipse callout,callout relative pointer={(-1,0.5)},text width=8cm]
    (2){ 
       所谓性质相同指的是数据元素具有相同数量和类型的数据项。 \vspace{0.1in}

       实际应用中,处理的数据元素通常具有相同性质,在不产生混淆的情况下,将数据对象简称为数据。     
    };

  \end{tikzpicture}  
\end{figure}

