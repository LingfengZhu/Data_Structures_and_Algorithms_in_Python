\begin{figure}
  
  \centering
  \begin{tikzpicture}[node distance=4.5cm]
    \node [fill=blue!20,draw,starburst,drop shadow,text width=7.5cm] (1) {
      \blue{\large 事后统计方法}\\
      通过设计好的测试程序和数据,利用计算机计时器对不同算法编制的程序的运行时间进行比较,从而确定效率的高低。
    };
    \pause 
    \node[below of=1,rectangle split,rectangle split parts=4,rounded corners=3mm,draw,text width=10cm,fill=red!20] {
      \blue{缺点}
      \nodepart{second}
      须依据算法事先编制好程序
      \nodepart{third}
      时间的比较依赖于软硬件等环境因素\\
      \begin{itemize}
      \item 不同性能的机器上算法的表现不尽相同;
      \item 不同操作系统、编译器等也会影响算法的运行结果;\\
      \item CPU使用率和内存占用情况也会造成微小差异。
      \end{itemize}
      \nodepart{fourth}
      测试数据设计困难,且程序运行时间还与测试数据的规模有很大关系,
      效率高的算法在小的测试数据面前往往得不到体现。
    };

  \end{tikzpicture}  
\end{figure}
