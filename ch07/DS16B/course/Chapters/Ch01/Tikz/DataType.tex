\begin{figure}  
  \centering
  \begin{tikzpicture}
    \node at (0,0)[left,fill=blue!20,draw,starburst,drop shadow,text=red,text centered]
    (1){
      数据类型
    } ;
    \node at (1.5,0)[right,fill=blue!20,draw,rectangle,rounded corners=3mm,text=blue,text width=6cm]
    (2){
      指一组性质相同的值的集合及定义在此集合上的一些操作的总称。
    };
    \pause
    \node at (2,-4) [fill=blue!20,draw,ellipse,text width=8cm]
    { \small
      在C语言中,按照取值的不同,数据类型可分为:\vspace{.1in}
      \begin{itemize}
      \item 原子类型:不可分解,包括整型、浮点型、字符型等;
      \item 由若干个类型组合而成,可再分解。如整型数组由若干个整型数据组成。
      \end{itemize}
    };

  \end{tikzpicture}
\end{figure}
