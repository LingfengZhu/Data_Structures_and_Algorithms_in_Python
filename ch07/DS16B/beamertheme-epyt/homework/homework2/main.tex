\documentclass[10pt,a4paper,twoside,openright,titlepage,fleqn,
tablecaptionabove]{article}

\usepackage{geometry}
\geometry{left=2.5cm,right=2.5cm,top=2.5cm,bottom=2.5cm}

\usepackage{amsmath,amssymb,amsthm}
\usepackage{courier}

\usepackage{extarrows}
\usepackage{verbatim,color,xcolor}
\usepackage{pgf}
\usepackage{tikz}
\usetikzlibrary{calc}
\usetikzlibrary{arrows,snakes,backgrounds,shapes}
\usetikzlibrary{matrix,fit,positioning,decorations.pathmorphing}
%% \usepackage{classicthesis}
\usepackage{CJK}
\usepackage{mathdots}

\usepackage{listings}
\lstset{
  keywordstyle=\color{blue!70},
  frame=single,
  basicstyle=\ttfamily\small,
  commentstyle=\small\color{red},
  breakindent=0pt,
  rulesepcolor=\color{red!20!green!20!blue!20},
  rulecolor=\color{black},
  tabsize=4,
  numbersep=5pt,
  breaklines=true,
  %% backgroundcolor=\color{red!10},
  showspaces=false,
  showtabs=false,
  extendedchars=false,
  escapeinside=``,
  frame=no,
}



\begin{document}

\begin{CJK}{UTF8}{gkai}

%%%% 设置标题
\title{第二次作业:栈(Stack)}
\author{李明(2014******)} %% 请在此处写上你的姓名和学号
\maketitle

%%%% 
\newtheorem*{jie}{{解}}
\newtheorem{wenti}{{问题}}



请完成以下问题。

%%%% 第一题 %%%%
\begin{wenti}
%%%%%
设共享栈的结构描述为:
\begin{lstlisting}[language=C]
#define MAXSIZE 100
#define ERROR 0
#define OK 1

typedef int ElemType;
typedef int Status;
typedef struct {
  ElemType data[MAXSIZE];
  int top1;
  int top2;
} Share_SqStack;
\end{lstlisting}
请编写能运行的完整程序,自行设计入栈、出栈过程,并输出相关结果。程序应包括以下函数:
\begin{lstlisting}[language=C]
Status Init(Share_SqStack * S)  //`初始化共享栈为空`
{
   ...
}
Status IsFull(Share_SqStack * S) //`判断共享栈是否满栈`
{
   ...
}

Status IsEmpty(Share_SqStack * S, int StackNumber) //`判断栈1或栈2是否为空`
{
   ...
}

Status Push(Share_SqStack * S, ElemType e, int StackNumber)
{
   ...
}

ElemType Pop(Share_SqStack * S, int StackNumber)
{
   ...
}

void PringS(Share_SqStack * S, int StackNumber) //`打印栈1或栈2`
{
   ...
}

Status Clear(Share_SqStack * S) //`将共享栈清空`
{
   ...
}

Status Destroy(Share_SqStack * S) //`将共享栈销毁`
{
   ...
}

int main(void)
{
  ...
}
\end{lstlisting}
返回链表的结点个数。
\end{wenti}


\begin{jie}

%%%%%%%%请将解答写在 \begin{jie} ... \end{jie}之间。建议在Visual Studio或XCode中把代码写好(代码放在code文件夹中),各函数分别写在一个单独的C文件中,文件名与函数名一致,然后用以下命令添加代码即可。

源代码:
\lstinputlisting[language=C]
{code/length.c}


运行结果:%(可将运行结果直接粘贴至以下环境,示例如下:)
\begin{lstlisting}
Initiation:
Stack #1: 
Stack #2:

After pushing 1 2 3 into Stack #1 and 4 5 6 into Stack #2:
Stack #1: 1 2 3
Stack #2: 4 5 6

After 1 pop in Stack #1 and 2 pop in Stack #2:
Stack #1: 1 2 
Stack #2: 4 

After clearing the sharing stack:
Stack #1:  
Stack #2:  
\end{lstlisting}

\end{jie}

%%%% 第2题 %%%%
\begin{wenti}
修改数值转换中的Convert函数,使其能处理十进制数到二、八、十六进制数的转换。
最后编写测试函数(即main函数),任意输入一个十进制数,输出二、八、十六进制数。
提示:可在Convert函数中设置字符数组
\begin{lstlisting}
char s[] = "0123456789ABCDEF";
\end{lstlisting}
在输出结果时用
\begin{lstlisting}
putc(s[*e]); 
\end{lstlisting}
\end{wenti}

\begin{jie}
源代码:% 可使用以下命令导入C文件(要求C文件放在code中)
\lstinputlisting[language=C]
{code/length.c}

运行结果:% 以下仅为示例
\begin{lstlisting} 
Enter a decimal number: 123
binary: 1111011    
octal: 173
hexadecimal: 7B
\end{lstlisting}
	
\end{jie}

 

\end{CJK}
\end{document}
