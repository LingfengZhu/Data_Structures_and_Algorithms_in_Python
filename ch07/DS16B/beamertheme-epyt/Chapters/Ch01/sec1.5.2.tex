\subsection{矩阵范数}

%%******
\begin{frame}\ft{\subsecname}

\begin{dingyi}[矩阵范数]
矩阵范数是一个$\|\cdot\|:\R^{n\times n}\to\R$的非负函数,它满足:\\
\begin{itemize}
\item[(1)] \textcolor{acolor4}{正定性}:
$$\|A\|\ge0, \quad \forall~A\in\R^{n\times n},  ~
\mbox{且}~
\|A\|=0 \Longleftrightarrow A=0;$$\\ 
\item[(2)] \textcolor{acolor4}{齐次性}:
$$\|\alpha A\|=|\alpha|\cdot \|A\|, \quad \forall~A\in\R^{n\times n},~\alpha\in\R;$$ 
\item[(3)] \textcolor{acolor4}{三角不等式}:
$$\|A+B\|\le \|A\|+\|B\|, \quad \forall~A, B\in\R^{n\times n};$$ 
\item[(4)] \textcolor{acolor4}{相容性}:
$$\|AB\|\le \|A\|\cdot \|B\|, \quad \forall~A, B\in\R^{n\times n}.$$
\end{itemize}
\end{dingyi}

\end{frame}


%%******
\begin{frame}\ft{\subsecname}

\begin{xingzhi}[矩阵范数的性质]
\begin{itemize}
\item[$1^{\circ}$]
$\R^{n\times n}$上的任意两个范数等价;\\[0.5cm]
\item[$2^{\circ}$]
矩阵序列的范数收敛等价于元素收敛,即
$$
\lim_{k\to\infty}\|A^{(k)}-A\|=0
\Longleftrightarrow
\lim_{k\to\infty} a_{ij}^{(k)} = a_{ij}, ~ i,j=1,\cd,n, 
$$
其中$A^{(k)} = (a_{ij}^{(k)}), ~A = (a_{ij})$。
\end{itemize}

\end{xingzhi}

\end{frame}

%%******
\begin{frame}\ft{\subsecname}\fst{矩阵范数与向量范数的相容性}

\begin{dingyi}[矩阵范数与向量范数的相容性]
若矩阵范数$\|\cdot\|_M$和向量范数$\|\cdot\|_v$满足
$$
\|Ax\|_v \le \|A\|_M \|x\|_v, \quad A\in\R^{n\times n},
\quad x\in\R^n,
$$
则称\textcolor{acolor5}{矩阵范数$\|\cdot\|_M$和向量范数$\|\cdot\|_v$相容}。
\end{dingyi}

\vspace{0.2cm}
\pause

\textcolor{acolor4}{若无特别说明,总假定矩阵范数和向量范数相容。}

\end{frame}

%%******
\begin{frame}\ft{\subsecname}\fst{从属范数}

\begin{dingli}
设$\|\cdot\|$是$\R^n$上的一个向量范数,若定义
$$
\VERT A\VERT = \max_{\|x\|=1}\|Ax\|,~A\in\R^{n\times n},
$$
则$\VERT\cdot\VERT$是$\R^{n\times n}$上的一个矩阵范数。
\end{dingli}
\vspace{0.2cm}
\pause

这样的矩阵范数$\VERT\cdot\VERT$称为\textcolor{acolor4}{从属于向量范数$\|\cdot\|$的矩阵范数},
也称\textcolor{acolor4}{由向量范数$\|\cdot\|$诱导出的算子范数}。

\end{frame}

%%******
\begin{frame}\ft{\subsecname}

\begin{zhengming}
1、先证$\VERT\cdot\VERT$有意义。\pause 
$$
\begin{array}{ll}
&\left.
\begin{array}{l}
\mathcal{D} = \{ x \in \R^n: ~ \|x\| =1 \} \mbox{是} \R^n\mbox{中有界闭集} \\[0.2cm] \pause
\|\cdot\|\mbox{是}\R^n\mbox{上的连续函数}
\end{array} \pause 
\right\} \\[0.4in] \pause 
\ds \Longrightarrow~~ & 
\mbox{存在}x_0\in \mathcal{D}\mbox{使得}\|Ax_0\| = \max_{x\in\mathcal{D}} \|Ax\|.
\end{array}
$$
\end{zhengming}

\end{frame}

%%******
\begin{frame}\ft{\subsecname}

\begin{zhengming}
2、$\forall 0 \ne x \in \R^n$,有
$$
\frac{\|Ax\|}{\|x\|} = \left\|A\frac{x}{\|x\|}\right\| \le \VERT A\VERT,         
$$
从而有
$$
\|Ax\| \le \VERT  A \VERT   \cdot \|x\|, \quad \forall x \in \R^n.
$$
\end{zhengming}

\end{frame}

%%******
\begin{frame}\ft{\subsecname}

\begin{zhengming}
3、验证$\VERT \cdot\VERT $满足矩阵范数的性质(1)-(3):\pause
\begin{itemize}
\item(正定性) 设$A\ne 0$,不妨设$A$的第$i$列非零,即$Ae_i\ne 0$,则
$$ 0 < \|A e_i\| \le \VERT A\VERT  \cdot \|e_i\|   
~~\Longrightarrow~~ 
\VERT  A \VERT  > 0. 
$$
\item\pause(齐次性)  $\forall \alpha \in \R$,
\[
\VERT \alpha A\VERT  = \max_{\|x\|=1} \| \alpha A x\| = |\alpha| \cdot \max_{\|x\|=1} \|Ax\| = |\alpha| \cdot \VERT A\VERT.
\]  
\item\pause(三角不等式) 设$x$满足$\|x\|=1$使得$\|(A+B)x\|=\VERT A+B\VERT $,则
$$
\begin{array}{rcl}
\VERT A+B\VERT  &=& \|(A+B)x\| \le \|Ax\| + \|Bx\| \\
&\le& \VERT A\VERT \cdot\|x\| + \VERT B\VERT \cdot\|x\| = \VERT A\VERT  + \VERT B\VERT .            
\end{array}
$$
\end{itemize}
\end{zhengming}

\end{frame}


%%******
\begin{frame}\ft{\subsecname}

\begin{proof}
4、验证$\VERT \cdot\VERT $满足矩阵范数的性质--相容性。
  设$x$满足$\|x\|=1$使得$\|ABx\|=\VERT AB\VERT $,则
$$
\begin{array}{rcl}
\VERT AB\VERT  &=& \|ABx\| \le \VERT A\VERT  \cdot \|Bx\| \\[0.2cm]
&\le& \VERT A\VERT \cdot \VERT B\VERT \cdot\|x\| = \VERT A\VERT  \cdot \VERT B\VERT .            
\end{array}
$$
\end{proof}

\end{frame}


%%******
\begin{frame}\ft{\subsecname}

由$\R^n$上的$p$范数可诱导出$\R^{n\times n}$上的算子范数$\|\cdot\|_p$:
$$
\|A\|_p = \max_{\|x\|_p=1}\|Ax\|_p, ~~ A \in \R^{n\times n}.
$$

\end{frame}


%%******
\begin{frame}\ft{\subsecname}


\begin{dingli}
设$A=(a_{ij})\in\R^{n\times n}$,则
\begin{itemize}
\item 
\textcolor{acolor5}{列范数} 
$$
\|A\|_1 = \max_{1\le j \le n} \sum_{i=1}^n |a_{ij}|
$$
\item
\textcolor{acolor5}{行范数}  
$$
\|A\|_{\infty} = \max_{1\le i \le n} \sum_{j=1}^n |a_{ij}|
$$
\item
\textcolor{acolor5}{谱范数}
$$
\|A\|_2 = \sqrt{\lambda_{max}(A^TA)}
$$
其中$\lambda_{max}(A^TA)$表示$A^TA$的最大特征值。
\end{itemize}

\end{dingli}

\end{frame}

%%******
\begin{frame}\ft{\subsecname}

\begin{zhengming}
$A=0$时定理显然成立,以下设$A\ne 0$。 \pause

\textcolor{acolor4}{(列范数)}设$A = [a_1, \cd, a_n], 
~\delta = \|a_{j_0}\|_1 = \max_{1\le j \le n}\|a_j\|_1$,则$\forall  x \in \R^n$且$\|x\|_1 = \sum_{i=1}^n |x_i| = 1$,有
$$
\begin{aligned}
\|Ax\|_1 & =\| \sum_{j=1}^n  x_j a_{j}\|_1 \le \sum_{j=1}^n |x_j| \|a_{j}\|_1  \\ 
&< \sum_{j=1}^n|x_j| \max_{1\le j \le n} \|a_j\|_1 = \|a_{j_0}\|_1 = \delta.
\end{aligned}
$$
因$\|e_{j_0}\|=1$且$\|Ae_{j_0}\|_1 = \|a_{j_0}\|_1 = \delta$
,故
$$
 \|A\|_1 = \max_{\|x\|_1=1}\|Ax\|_1 = \delta = \max_{1\le j \le n} \|a_j\|_1 = \max_{1\le j \le n} \sum_{i=1}^n|a_{ij}|.
$$
\end{zhengming}
\end{frame}

%%******
\begin{frame}\ft{\subsecname}

\begin{zhengming}
\textcolor{acolor4}{(行范数)}令$\eta = \max_{1\le j \le n} \sum_{j=1}^n |a_{ij}|$,则$\forall  x \in \R^n$且 $\|x\|_{\infty}  = 1$有
$$
\|Ax\|_{\infty} =   \max_{1\le i \le n}| \sum_{j=1}^n   a_{ij} x_j| 
\le \max_{1\le i \le n} \sum_{j=1}^n |a_{ij}||x_j|   
\le \ds  \max_{1\le i \le n}  \sum_{j=1}^n |a_{ij}|  = \eta.
$$
\pause 
设$A$的第$k$行的1范数最大,即$\eta = \max_{j=1}^n |a_{kj}|$。令
$$
\tilde{x} = \left(\mathrm{sgn}(a_{k1}),~\cd,~\mathrm{sgn}(a_{kn})\right)^T,
$$
则 \pause 
$$
\begin{array}{l}
A \ne 0 ~\Longrightarrow~ \|\tilde{x}\|_{\infty} = 1 
~\Longrightarrow~ \|A\tilde{x}\|_{\infty} = \eta
\end{array}
$$ 
\pause
从而
$$
\|A\|_{\infty} = \eta =  \max_{1\le j \le n} \sum_{j=1}^n |a_{ij}|.
$$
\end{zhengming}
\end{frame}

%%******
\begin{frame}\ft{\subsecname}

\begin{zhengming}
\textcolor{acolor4}{(谱范数)}注意到
$$\textcolor{acolor5}{\|A\|_2 = \max_{\|x\|_2=1} \|Ax\|_2  = \max_{\|x\|_2=1} \sqrt{x^T(A^TA)x}}.
$$  
考察$\textcolor{acolor5}{A^TA}$,因其对称半正定,可设其特征值为
$\lambda_1 \ge \lambda_2 \ge \cd \ge \lambda_n \ge 0$,特征向量对应为$v_1, \cd, v_n \in \R^n$且$\|v_i\|=1$。\end{zhengming}

\end{frame}

%%******
\begin{frame}\ft{\subsecname}
\begin{proof} \textcolor{acolor4}{(谱范数)}
一方面,$\forall x \in \R^n\mbox{且} \|x\|_2 = 1$有$ 
x = \sum_{i=1}^n \alpha_i v_i$和$\sum_{i=1}^n \alpha_i^2 = 1$,从而
$$
x^T A^T A x = \sum_{i=1}^n \lambda_i \alpha_i^2 \le \lambda_1;
$$
\pause
另一方面,取$x=v_1$,则有
$$
x^T A^T A x = v_1^T A^T A v_1 = v_1^T \lambda_1 v_1 = \lambda_1.
$$
\pause
于是
$
\|A\|_{2} =  \max_{\|x\|_2=1} \|Ax\|_2 = \sqrt{\lambda_1} = \sqrt{\lambda_{max}(A^TA)}.
$
\end{proof}

\end{frame}

%%******
\begin{frame}\ft{\subsecname}\fst{谱范数的性质}

\begin{dingli}
设$A  \in \R^{n\times n}$,则有
\begin{itemize}
\item[(1)]
$$
\|A\|_2 = \max\left\{|y^TAx|: ~x, y \in \mathbf{C}^n,~ \|x\|_2 = \|y\|_2 = 1\right\}
$$
\item[(2)]
$$
\|A^T\|_2 = \|A\|_2 = \sqrt{\|A^TA\|_2}
$$
\item[(3)]      
$$
\|UA\|_2=\|AV\|_2=\|A\|_2, \quad \forall\mbox{正交矩阵}U, V
$$
\end{itemize}
\end{dingli}

\end{frame}

%%******
\begin{frame}\ft{\subsecname}

\begin{dingyi}[Frobenius范数]
$$\textcolor{acolor4}{
\|A\|_F = \sqrt{\sum_{i,j=1}^n|a_{ij}|^2}
}
$$
\end{dingyi}
它是向量2范数的自然推广。

\end{frame}



%%******
\begin{frame}\ft{\subsecname}

\begin{dingyi}[谱半径]
设$A\in \mathbf{C}^{n\times n}$,则称
$$\textcolor{acolor4}{
\rho(A)=\max\{|\lambda|:~\lambda\in\lambda(A)\}
}
$$
为$A$的谱半径,其中$\lambda(A)$表示$A$的特征值的全体。
\end{dingyi}

\end{frame}

%%******
\begin{frame}\ft{\subsecname}\fst{谱半径与矩阵范数的关系}

\begin{dingli}
设$A\in \mathbf{C}^{n\times n}$,则
\begin{itemize}
\item[(1)]
对$\mathbf{C}^{n\times n}$上的任意矩阵范数$\|\cdot\|$,有
$$ \textcolor{acolor4}{
\rho(A) \le \|A\|
}
$$
\item[(2)]
$\forall \epsilon>0$,存在$C^{n\times n}$上的矩阵范数$\|\cdot\|$使得
$$ \textcolor{acolor4}{
\|A\| \le \rho(A) + \epsilon} 
$$
\end{itemize}•
\end{dingli}

\end{frame}


%%******
\begin{frame}\ft{\subsecname}

\begin{zhengming}
1、设$A\in \mathbf{C}^{n\times n}$满足
$$
x \ne 0, \quad Ax = \lambda x, \quad |\lambda| = \rho(A),
$$ \pause
则有
$$
\rho(A) \|xe_1^T\| = \|\lambda x e_1^T\| = \|A x e_1^T\| \le \|A\| \cdot \|x e_1^T\|,
$$
从而
$$
\rho(A) \le \|A\|.
$$
\end{zhengming}
\end{frame}


%%******
\begin{frame}\ft{\subsecname}

\begin{zhengming}
2、由Jordan分解定理知,存在非奇异阵$X\in \mathbf{C}^{n\times n}$,使得
$$
X^{-1} A X = \left[
\begin{array}{ccccc}
\lambda_1 & \delta_1 & & & \\
& \lambda_2 & \delta_2 & & \\
& & \dd & \dd &\\
& & & \lambda_{n-1} & \delta_{n-1}  \\
& & &  & \lambda_{n} 
\end{array}
\right], \quad \delta_i = 1 \mbox{或} 0.
$$ 
\end{zhengming}
\end{frame}


%%******
\begin{frame}\ft{\subsecname}

\begin{zhengming}
$\forall \epsilon > 0$,令$D_{\epsilon} = \mathrm{diag} (1, \epsilon, \cd, \epsilon^{n-1})$,有
$$
D_{\epsilon}^{-1}X^{-1} A X D_{\epsilon} = \left[
\begin{array}{ccccc}
\lambda_1 & \epsilon\delta_1 & & & \\
& \lambda_2 & \epsilon\delta_2 & & \\
& & \dd & \dd &\\
& & & \lambda_{n-1} & \epsilon\delta_{n-1}  \\
& & &  & \lambda_{n} 
\end{array}
\right].
$$
定义
$$
\|G\|_{\epsilon} = \| D_{\epsilon}^{-1}X^{-1} G X D_{\epsilon}  \|_{\infty}, \quad G \in \mathbf{C}^{n\times n},
$$
则$\|\cdot\|_{\epsilon}$是由向量范数
$\textcolor{acolor5}{
\|x\|_* = \| (XD_{\epsilon})^{-1} x \|_{\infty}
}
$
诱导出的算子范数。
\end{zhengming}

\end{frame}


%%******
\begin{frame}\ft{\subsecname}

\begin{proof}
事实上,
$$
\begin{array}{rcl}
\ds \|G\|_{\epsilon} &=&  \ds  \max_{\|y\|_{\infty}=1} \| D_{\epsilon}^{-1}X^{-1} G X D_{\epsilon}  y\|_{\infty}\\
&\xlongequal[]{y = (XD_{\epsilon})^{-1}x}&\ds
\max_{\|\textcolor{acolor5}{(XD_{\epsilon})^{-1}x\|_{\infty} =1}}\| D_{\epsilon}^{-1}X^{-1} G  x\|_{\infty} \\[0.4cm]
&=& \ds \max_{\textcolor{acolor5}{\|x\|_{*} =1}}\| G  x\|_{*} .
\end{array}
$$
另外,还有
$$
\|A\|_{\epsilon} = \| D_{\epsilon}^{-1}X^{-1} A X D_{\epsilon}  \|_{\infty}
= \max_{1\le i \le n} (|\lambda_i + |\epsilon \delta_i|) \le \rho(A) + \epsilon.
$$
\end{proof}


\end{frame}



%%******
\begin{frame}\ft{\subsecname}

\begin{dingli}
设$A\in C^{n\times n}$,则
$$ 
\lim_{k\to\infty} A^k =0 \Longleftrightarrow \rho(A)<1
$$
\end{dingli}  \pause
\begin{proof}
先证必要性:
$$
\begin{array}{rl}
\left.
\begin{array}{l}
\lim_{k\to\infty} A^k =0 \\[0.2cm] \pause 
|\lambda_0| = \rho(A) \pause 
\end{array}
\right\}\pause
&~~\Longrightarrow ~~
\rho(A)^k = |\lambda_0|^k \le \rho(A^k) \le \|A^k\|_2( \forall k) \\[0.1in]
& \pause 
~~\Longrightarrow ~~
\rho(A) < 1.
\end{array}
$$
再证充分性:  
$$
\begin{array}{rcl}
\rho(A) < 1  
&~~\Longrightarrow ~~& \pause 
\mbox{必有算子范数}\|\cdot\|\mbox{使得}\|A\|<1 \\[0.3cm]  
&~~\Longrightarrow ~~& \pause 
0 \le \|A^k \| \le \|A\|^k \rightarrow 0, \quad k \rightarrow \infty.
\end{array}
$$
\end{proof}


\end{frame}


%%******
\begin{frame}\ft{\subsecname}

\begin{dingli}
设$A\in \mathbf{C}^{n\times n}$,则
\begin{itemize}
\item[(1)]
$$
\sum_{k=0}^{\infty}A^k\mbox{收敛}
~~\Longleftrightarrow~~
\rho(A)<1
$$
\pause
\item[(2)]
$$
\sum_{k=0}^{\infty}A^k\mbox{收敛} 
~~\Longleftrightarrow~~
\left\{
\begin{array}{l}
\ds \sum_{k=0}^{\infty} A^k = (I-A)^{-1}, \\[0.8cm] 
\ds  \mbox{存在}\mathbf{C}^{n\times n}\mbox{上的矩阵范数}\|\cdot\|\mbox{使得}\forall m \in \mathbf{N} \\[0.4cm]
\ds  \qquad\left\|(I-A)^{-1}-\sum_{k=0}^{m} A^k\right\|
\le \frac{\|A\|^{m+1}}{1-\|A\|}
\end{array}
\right.
$$
\end{itemize}    
\end{dingli}

\end{frame}

%%******
\begin{frame}\ft{\subsecname}

\begin{zhengming}
1、\textcolor{acolor5}{先证必要性:}记$A^k = (a_{ij}^{(k)})$,则
$$
\sum_{k=0}^{\infty}A^k = I + A + A^2 + \cd + A^k + \cd,
$$
其元素为数项级数
$$
\delta_{ij} + a_{ij}^{(1)} + a_{ij}^{(2)} + \cd + a_{ij}^{(k)} + \cd,
$$
而数项级数收敛的必要条件是$a_{ij}^{(k)} \to 0, \quad k \to \infty$。\pause 故
$$
\begin{array}{rcl}
\sum_{k=0}^{\infty}A^k\mbox{收敛}
&~~\Longrightarrow~~& 
\ds\lim_{m\to\infty} a_{ij}^{(m)}  = 0 \pause 
~~\Longrightarrow~~
\lim_{m\to\infty} A^m  = 0   \\[0.3cm] \pause
&\pause \Longrightarrow~~& \pause 
\rho(A) < 1.
\end{array}
$$
\end{zhengming}
\end{frame}

%%******
\begin{frame}\ft{\subsecname}
\begin{zhengming}
1、\textcolor{acolor5}{再证充分性:}
$$
\begin{array}{rcl}          
\rho(A) < 1
&~~\Longrightarrow~~&
\lim_{m\to\infty} A^m  = 0 \textcolor{acolor5}{\mbox{且}I-A\mbox{可逆}}\\[0.3cm] 
&~~\Longrightarrow~~&
\lim_{m\to\infty}(I-A)\sum_{k=0}^m A^k  = \lim_{m\to\infty} (I - A^m) = I \\[0.3cm]
&~~\Longrightarrow~~&
\sum_{k=0}^{\infty}A^k\mbox{收敛,且}(I-A)^{-1} = \sum_{k=0}^m A^k.
\end{array}
$$
\end{zhengming}

\end{frame}

%%******
\begin{frame}\ft{\subsecname}

\begin{zhengming}
2、
$$
\begin{array}{rcl}
(I-A)^{-1} - \sum_{k=0}^{m}A^k &~~= ~~&  \sum_{k=m+1}^{\infty}A^k \\[0.4cm]
~~\Longrightarrow~~
\left\| (I-A)^{-1} - \sum_{k=0}^{m}A^k\right\| 
&~~= ~~&  \left\|\sum_{k=m+1}^{\infty}A^k\right\| \\[0.4cm]
&~~\le~~&  \sum_{k=m+1}^{\infty}\|A\|^k = \ds \frac{\|A\|^{m+1}}{1-\|A\|}
\end{array}
$$
\end{zhengming}

\end{frame}

%%******
\begin{frame}\ft{\subsecname}

\begin{tuilun}
设$A\in \mathbf{C}^{n\times n}$,$\mathbf{C}^{n\times n}$上的$\|\cdot\|$满足$\|I\|=1$,则
$$
\|A\|<1
~~\Longrightarrow~~ \pause 
\left\{
\begin{array}{l}
\ds \textcolor{acolor5}{I-A\mbox{可逆}} \\[0.3cm]
\pause 
\ds \left\|(I-A)^{-1}\right\|
\le \frac1{1-\|A\|}
\end{array}
\right.
$$
\end{tuilun}

\end{frame}

