\subsection{误差与有效数字}

\subsubsection{绝对误差与绝对误差限}
%*******
\begin{frame}\ft{\subsecname}

\begin{dingyi}[绝对误差与绝对误差限]
设某个量的精确值为$x$,其近似值为$x^{\star}$,则称
$$
E(x) = x - x^{\star}
$$
为近似值$x^{\star}$的\textcolor{acolor5}{绝对误差},简称\textcolor{acolor5}{误差}。
若存在$\eta>0$,使得
$$
|E(x)=|x - x^{\star}|\le\eta
$$
则称$\eta$为近似值$x^{\star}$的\textcolor{acolor5}{绝对误差限},简称\textcolor{acolor5}{误差限}或\textcolor{acolor5}{精度}。
\end{dingyi}

\rule{\textwidth}{.5mm}
$\eta$越小,表示近似值$x^{\star}$的精度越高。
$$
x^{\star} - \eta \le x \le x^{\star} +  \eta
~~\mbox{或}~~
x = x^{\star} \pm \eta
$$

\end{frame}


%*******
\begin{frame}\ft{\subsecname} 



\begin{li}
用毫米刻度的直尺量一长度为$x$的物体,测得其近似值为$x^{\star}=84$mm。
\end{li}
\vspace{0.3cm}
\pause

因直尺以mm为刻度,其误差不超过$0.5$mm,即有
$$
|x-84|\le0.5 \mbox{~mm}
~~\mbox{或}~~
x=84\pm0.5\mbox{~mm}
$$

\end{frame}

%*******
\begin{frame}\ft{\subsecname}\fst{\subsubsecname}

\begin{li}
测量$100$m和$10$m的两个长度,若它们的绝对误差均为$1$cm,显然前者的测量更为精确。
\end{li}

\pause
\vspace{0.3cm}
由此可见,决定一个量的近似值的精确度,\textcolor{acolor3}{除了绝对误差外,还必须考虑该量本身的大小},为此引入相对误差的概念。

\end{frame}

%*******
\begin{frame}\ft{\subsecname}\fst{\subsubsecname}


\begin{dingyi}[相对误差与相对误差限]
近似值$x^{\star}$的\textcolor{acolor5}{相对误差}是绝对误差与精确值之比,即
$$
E_r(x) = \frac{E(x)}{x} = \frac{x-x^{\star}}{x}.
$$
%\pause
实际中精确值$x$一般无法知道,通常取
$$\textcolor{acolor5}{
E_r(x) = \frac{E(x)}{x^{\star}} = \frac{x-x^{\star}}{x^{\star}}.
}
$$
%\pause
若存在$\delta>0$,使得
$$
|E_r(x)|=|\frac{x-x^{\star}}{x^{\star}}|\le\delta,
$$
则称$\delta$为近似值$x^{\star}$的\textcolor{acolor5}{相对误差限}。
\end{dingyi}


\end{frame}



%*******
\begin{frame}\ft{\subsecname}\fst{\subsubsecname}

\begin{li}
当$|x-x^{\star}|\le 1$cm时,测量$100$m物体时的相对误差为
$$
|E_r(x)|=\frac{1}{10000}=0.01\%,
$$
测量$10$m物体时的相对误差为
$$
|E_r(x)|=\frac{1}{1000}=0.1\%.
$$
\end{li}


\end{frame}
 
%*******
\begin{frame}[fragile]\ft{\subsecname} 

\begin{dingyi}[有效数字]
若近似值$x^{\star}$的绝对误差限是某一位的半个单位,就称其精确到这一位,
且从该位直到$x^{\star}$的第一位非零数字共有$n$位,则称近似值$x^{\star}$有$n$位有效数字。
\end{dingyi}

\pause
\begin{table}
\begin{tabular}{|l|l|}\hline
给定数& 具有5位有效数字的近似值 \\ \hline
$358.467$            & $358.47$\\
$0.00427511$         & $0.0042751$\\
\textcolor{acolor5}{$8.000034$}  & \textcolor{acolor5}{$8.0000$}\\
$8.000034\times10^3$ & $8000.0$\\\hline
\end{tabular}
\end{table}

\end{frame}
%
%
%
%*******
\begin{frame}\ft{\subsecname} 


任何一个实数$x$经四舍五入后得到的近似值$x^{\star}$都可写成
$$\textcolor{acolor3}{
x^{\star} = \pm0.\alpha_1\alpha_2\cd\alpha_n\times 10^m.
}
$$
当其绝对误差限满足
$$
|x-x^{\star}|\le \frac12 \times 10^{m-n}
$$
时,则称近似值$x^{\star}$具有$n$位有效数字,其中$m$为整数,
$\alpha_1$为1到9中的一个数字,$\alpha_1,\cdots,\alpha_n$是0到9中的数字。


\end{frame}



%*******
\begin{frame}\ft{\subsecname}\fst{\subsubsecname}

\begin{li}
验证$e=2.718281828459046\cd$的近似值$2.71828$具有$6$位有效数字
\end{li}
\begin{jie}
$$
\begin{array}{ll}
&2.71828=0.271828\times 10  \\[0.1in]
\pause
&~~ \Longrightarrow~~ m =1 , n = 6. \\[0.2in]
\pause
&\because
|e-2.71828| = 0.000001828\cd < \frac12 \times 10^{-5}, \\[0.2cm]
\pause
&\therefore
\mbox{它具有$6$位有效数字}
\end{array}
$$
\end{jie}

\end{frame}


%%%%%%%%%%%%%%%%% 
\begin{frame}\ft{\subsecname} 
\begin{jielun}[1]
设某数$x$的近似值为$x^{\star}$,则
$$
x^{\star}\mbox{有}n\mbox{位有效数字}
~~\Longrightarrow~~
|x-x^{\star}|\le \frac12 \times 10^{m-n}.
$$
\end{jielun}    

\pause 
\rule{\textwidth}{0.3mm}
\textcolor{acolor5}{当$m$一定时,有效数字位数$n$越多,其绝对误差限越小。}


\end{frame}



%*******
\begin{frame}\ft{\subsecname}\fst{\subsubsecname}


\begin{jielun}[2]
设某数$x$的近似值$x^{\star}$形如
$$
x^{\star} = \pm0.\alpha_1\alpha_2\cd\alpha_n \times 10^m,
$$
则
\begin{itemize}
\item
$ 
x^{\star}\mbox{有}n\mbox{位有效数字}
~\Rightarrow~
E_r^{\star}(x)\le \frac1{2\alpha_1}\times 10^{-(n-1)};
$\\[0.2in]
\item
$ 
E_r^{\star}(x)\le \frac1{2(\alpha_1+1)}\times 10^{-(n-1)}
~\Rightarrow~
x^{\star}\mbox{至少有}n\mbox{位有效数字}.
$
\end{itemize}
\end{jielun}

\pause
\rule{\textwidth}{0.3mm}

\textcolor{acolor5}{有效数字的位数越多,相对误差限就越小。}

\end{frame}

