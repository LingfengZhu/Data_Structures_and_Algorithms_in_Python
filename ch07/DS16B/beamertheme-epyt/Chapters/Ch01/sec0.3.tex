\section{0.3.~矩阵分解是设计算法的主要技巧}

%%%%%%%%%%
\begin{frame}\ft{\secname}
  \begin{small}
    \begin{block}{设计算法的基本思想}
      设法将一个一般的矩阵计算问题转化为一个或多个易于求解的问题,
      最主要的技巧就是\blue{矩阵分解},
      即\blue{将一个给定矩阵分解为几个特殊类型的矩阵的乘积。}
    \end{block}
  \end{small}
\end{frame}


%%%%%%%%%%
\begin{frame}\ft{\secname}
  \begin{small}
    \begin{li}
      $$
      \left[
        \begin{array}{cccc}
          l_{11}  &         &        & \\
          l_{21}  & l_{22}   &        & \\
          \vd    & \vd     & \dd    & \\
          l_{n1}  & l_{n2}   & \cd    &  l_{nn}\\
        \end{array}
        \right] \left[
        \begin{array}{c}
          x_{1}  \\
          x_{2}  \\
          \vd \\
          x_{n}  
        \end{array}
        \right] = \left[
        \begin{array}{c}
          b_{1}  \\
          b_{2}  \\
          \vd   \\
          b_{n}  
        \end{array}
        \right]
        $$
        和
        $$
        \left[
        \begin{array}{ccccc}
          u_{11} & u_{12}  & \cd & u_{1n} \\
                & u_{22} & \cd & u_{1n} \\
                &        & \dd & \vd \\
                &        &     & u_{nn} \\
        \end{array}
        \right] \left[
        \begin{array}{c}
          y_{1}  \\
          y_{2}  \\
          \vd \\
          y_{n}  
        \end{array}
        \right] = \left[
        \begin{array}{c}
          c_{1}  \\
          c_{2}  \\
          \vd   \\
          c_{n}  
        \end{array}
        \right]
      $$
      易于求解。
      \end{li}
      \end{small}
\end{frame}


%%%%%%%%%%
\begin{frame}\ft{\secname}
  \begin{small}
  \begin{li}
   对于一般的线性方程组$Ax=b$,可先将$A$作分解
      $$
      \begin{array}{l}
        PA=LU \quad
        P \to \mbox{排列矩阵} \quad
        L \to \mbox{下三角矩阵} \quad
        U \to \mbox{上三角矩阵} 
      \end{array}      
      $$
      \pause
      然后再求
      $$
      Ly = Pb, \quad Ux = y
      $$
      可得原方程组的解。
    \end{li}
  \end{small}
\end{frame}
