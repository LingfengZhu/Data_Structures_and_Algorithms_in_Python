\subsection{顺序栈}

% \begin{frame}\ft{\subsecname}
%   既然栈是线性表的特例,那么栈的顺序存储其实是线性表顺序存储的简化。
%   栈的顺序存储结构简称顺序栈,。根据数组是否可以根据需要增大,又可分为静态顺序栈和动态顺序栈。
% \begin{itemize}
% \item[$\diamond$]
% 静态顺序栈实现简单,但不能根据需要增大栈的存储空间;
% \item[$\diamond$]
% 动态顺序栈可以根据需要增大栈的存储空间,但实现稍微复杂。
% \end{itemize}

% \end{frame}

\begin{frame}\ft{\subsecname}
  既然栈是线性表的特例,那么栈的顺序存储其实是线性表顺序存储的简化。
  栈的顺序存储结构简称顺序栈,用数组来实现。 \vskip.1in

  \pause
  \begin{wenti}
    对于栈,用数组哪一端作为栈顶或栈底会比较好?
  \end{wenti} \vskip.1in

  \pause
  下标为0的一端作为栈底比较好,因为首元素都存在栈底,变化最小。
\end{frame}
%
%
\begin{frame}[fragile]\ft{\subsecname}
  \begin{lstlisting}[title=栈的结构定义,language=C]
    #define MAXSIZE 100
    typedef int ElemType;
    typedef struct SqStack{
      ElemType data[MAXSIZE];
      int top;
    }SqStack;
  \end{lstlisting}

  \begin{itemize}
  \item {\tt top}用于指示栈顶元素在数组中的位置,它必须小于存储栈的长度{\tt StackSize};
  \item 当栈存在一个元素时,{\tt top == 0},因此通常把空栈的判定条件定为{\tt top == -1}.
  \end{itemize}
\end{frame}
%
%
\begin{frame}\ft{\subsecname}
  \begin{figure}
  \centering
  \begin{tikzpicture}
    \tikzstyle{information text}=[fill=white,inner sep=1ex]
    \def\x{1.5}
    \def\y{0.7}


    \def\r{0}    
    \draw[very thick] (\r,5*\y)--(\r,0*\y)--(\r+\x,0*\y)--(\r+\x,5*\y);
    \node[below,style=information text]at(\r+0.5*\x,-1*\y){有两个元素,{\tt top=1}};
    \foreach \c in {1,2,...,5}{
      \edef\cc{\number\numexpr\c-1\relax}
      \node[left] at (\r+0*\x,\c*\y-0.5*\y) {\cc};
      \ifthenelse{\c=1 \OR \c=2}{
        \node [] at (\r+0.5*\x,\c*\y-0.5*\y) {$a_\c$};
        \filldraw[fill=red!40,fill opacity=0.5] (\r+0.05*\x,\c*\y-0.95*\y)rectangle(\r+0.95*\x,\c*\y-0.05*\y);
        \ifthenelse{\c=2}{          
          \draw[->,>=stealth,very thick] (\r-\x,\c*\y-0.5*\y) --node[above] {{\tt top}}(\r-0.4*\x,\c*\y-0.5*\y);
        }{}
      }{
        \filldraw[fill=white,fill opacity=0.5] (\r+0.05*\x,\c*\y-0.95*\y)rectangle(\r+0.95*\x,\c*\y-0.05*\y);     }
    }

    \pause 
    %% 
    \def\r{4}    
    \draw[very thick] (\r,5*\y)--(\r,0*\y)--(\r+\x,0*\y)--(\r+\x,5*\y);
    \node[below,style=information text]at(\r+0.5*\x,-1*\y){空栈,{\tt top=-1}};
    \foreach \c in {1,2,...,5}{
      \edef\cc{\number\numexpr\c-1\relax}
      \node[left] at (\r+0*\x,\c*\y-0.5*\y) {\cc};
      \filldraw[fill=white,fill opacity=0.5] (\r+0.05*\x,\c*\y-0.95*\y)rectangle(\r+0.95*\x,\c*\y-0.05*\y); 
    }
    \def\c{0}
    \draw[->,>=stealth,very thick] (\r-\x,\c*\y-0.5*\y) --node[above] {{\tt top}}(\r-0.4*\x,\c*\y-0.5*\y);


    \pause 
    %% 
    \def\r{8}    
    \draw[very thick] (\r,5*\y)--(\r,0*\y)--(\r+\x,0*\y)--(\r+\x,5*\y);
    \node[below,style=information text]at(\r+0.5*\x,-1*\y){栈满,{\tt top=4}};
    \foreach \c in {1,2,...,5}{
      \edef\cc{\number\numexpr\c-1\relax}
      \node[left] at (\r+0*\x,\c*\y-0.5*\y) {\cc};
      \node [] at (\r+0.5*\x,\c*\y-0.5*\y) {$a_\c$};
      \filldraw[fill=red!40,fill opacity=0.5] (\r+0.05*\x,\c*\y-0.95*\y)rectangle(\r+0.95*\x,\c*\y-0.05*\y);
    }
    \def\c{5}
    \draw[->,>=stealth,very thick] (\r-\x,\c*\y-0.5*\y) --node[above] {{\tt top}}(\r-0.4*\x,\c*\y-0.5*\y);



  \end{tikzpicture}
\end{figure}

\end{frame}
%
\begin{frame}\ft{\subsecname}
  \begin{figure}
  \centering
  \begin{tikzpicture}
    \tikzstyle{information text}=[fill=white,inner sep=1ex]
    \def\x{1.5}
    \def\y{0.7}


    \def\r{0}    
    \draw[very thick] (\r,5*\y)--(\r,0*\y)--(\r+\x,0*\y)--(\r+\x,5*\y);
    \node[below,style=information text]at(\r+0.5*\x,-1*\y){数组{\tt data}};
    \foreach \c in {1,2,...,5}{
      \edef\cc{\number\numexpr\c-1\relax}
      \node[left] at (\r+0*\x,\c*\y-0.5*\y) {\cc};
      \ifthenelse{\c=1 \OR \c=2}{
        \node [] at (\r+0.5*\x,\c*\y-0.5*\y) {$a_\c$};
        \filldraw[fill=red!40,fill opacity=0.5] (\r+0.05*\x,\c*\y-0.95*\y)rectangle(\r+0.95*\x,\c*\y-0.05*\y);
        \ifthenelse{\c=2}{          
          \draw[->,>=stealth,very thick] (\r-\x,\c*\y-0.5*\y) --node[above] {{\tt top}}(\r-0.4*\x,\c*\y-0.5*\y);
        }{}
      }{
        \filldraw[fill=white,fill opacity=0.5] (\r+0.05*\x,\c*\y-0.95*\y)rectangle(\r+0.95*\x,\c*\y-0.05*\y);     }
    }

    \pause 
    %% 
    \def\r{4}    
    \draw[very thick] (\r,5*\y)--(\r,0*\y)--(\r+\x,0*\y)--(\r+\x,5*\y);
    \node[below,style=information text]at(\r+0.5*\x,-1*\y){元素{\tt e}进栈};
    \foreach \c in {1,2,...,5}{
      \edef\cc{\number\numexpr\c-1\relax}
      \node[left] at (\r+0*\x,\c*\y-0.5*\y) {\cc};
      \ifthenelse{\c=1 \OR \c=2 \OR \c=3}{
        \ifthenelse{\c=3}{
          \node [] at (\r+0.5*\x,\c*\y-0.5*\y) {$e$};
          \filldraw[fill=blue!40,fill opacity=0.5] (\r+0.05*\x,\c*\y-0.95*\y)rectangle(\r+0.95*\x,\c*\y-0.05*\y);
          \draw[->,>=stealth,very thick] (\r-\x,\c*\y-0.5*\y) --node[above] {{\tt top}}(\r-0.4*\x,\c*\y-0.5*\y);
        }{
          \node [] at (\r+0.5*\x,\c*\y-0.5*\y) {$a_\c$};
          \filldraw[fill=red!40,fill opacity=0.5] (\r+0.05*\x,\c*\y-0.95*\y)rectangle(\r+0.95*\x,\c*\y-0.05*\y);
        }
      }{
        \filldraw[fill=white,fill opacity=0.5] (\r+0.05*\x,\c*\y-0.95*\y)rectangle(\r+0.95*\x,\c*\y-0.05*\y);     }
    }


  \end{tikzpicture}
\end{figure}

\end{frame}
%
\begin{frame}[fragile]\ft{\subsecname}
  \lstinputlisting[
    title={\tt Push.c},
    language=C,
  ]{Chapters/Ch03/Code/SqStack/Push.c}
\end{frame}

\begin{frame}\ft{\subsecname}
  \begin{figure}
\centering
\begin{tikzpicture}
\tikzstyle{information text}=[fill=white,inner sep=1ex]
\def\x{1.5}
\def\y{0.7}



%% 
\def\r{0}    
\draw[very thick] (\r,5*\y)--(\r,0*\y)--(\r+\x,0*\y)--(\r+\x,5*\y);
\node[below,style=information text]at(\r+0.5*\x,-1*\y){数组{\tt data}};
\foreach \c in {1,2,...,5}{
\edef\cc{\number\numexpr\c-1\relax}
\node[left] at (\r+0*\x,\c*\y-0.5*\y) {\cc};
\ifthenelse{\c=1 \OR \c=2 \OR \c=3}{
\ifthenelse{\c=3}{
  \node [] at (\r+0.5*\x,\c*\y-0.5*\y) {$e$};
  \filldraw[fill=blue!40,fill opacity=0.5] (\r+0.05*\x,\c*\y-0.95*\y)rectangle(\r+0.95*\x,\c*\y-0.05*\y);
  \draw[->,>=stealth,very thick] (\r-\x,\c*\y-0.5*\y) --node[above] {{\tt top}}(\r-0.4*\x,\c*\y-0.5*\y);
}{
  \node [] at (\r+0.5*\x,\c*\y-0.5*\y) {$a_\c$};
  \filldraw[fill=red!40,fill opacity=0.5] (\r+0.05*\x,\c*\y-0.95*\y)rectangle(\r+0.95*\x,\c*\y-0.05*\y);
}
}{
\filldraw[fill=white,fill opacity=0.5] (\r+0.05*\x,\c*\y-0.95*\y)rectangle(\r+0.95*\x,\c*\y-0.05*\y);     }
}

\pause
\def\r{4}    
\draw[very thick] (\r,5*\y)--(\r,0*\y)--(\r+\x,0*\y)--(\r+\x,5*\y);
\node[below,style=information text]at(\r+0.5*\x,-1*\y){元素{\tt e}出栈};
\foreach \c in {1,2,...,5}{
\edef\cc{\number\numexpr\c-1\relax}
\node[left] at (\r+0*\x,\c*\y-0.5*\y) {\cc};
\ifthenelse{\c=1 \OR \c=2}{
\node [] at (\r+0.5*\x,\c*\y-0.5*\y) {$a_\c$};
\filldraw[fill=red!40,fill opacity=0.5] (\r+0.05*\x,\c*\y-0.95*\y)rectangle(\r+0.95*\x,\c*\y-0.05*\y);
\ifthenelse{\c=2}{          
  \draw[->,>=stealth,very thick] (\r-\x,\c*\y-0.5*\y) --node[above] {{\tt top}}(\r-0.4*\x,\c*\y-0.5*\y);
}{}
}{
\filldraw[fill=white,fill opacity=0.5] (\r+0.05*\x,\c*\y-0.95*\y)rectangle(\r+0.95*\x,\c*\y-0.05*\y);     }
}



\end{tikzpicture}
\end{figure}

\end{frame}

\begin{frame}[fragile]\ft{\subsecname}
  \lstinputlisting[
    title={\tt Pop.c},
    language=C,
  ]{Chapters/Ch03/Code/SqStack/Pop.c}
\end{frame}


\begin{frame}[fragile]
\begin{center}
  \textcolor{acolor5}{\Large 顺序栈之完整程序}
\end{center}
\end{frame}
\begin{frame}[fragile,allowframebreaks]\ft{\tt SqStack.h}
  \lstinputlisting[
    language=C,
  ]{Chapters/Ch03/Code/SqStack/SqStack.h}
\end{frame}

\begin{frame}[fragile]\ft{\tt Init.c}
  \lstinputlisting[
    language=C,
  ]{Chapters/Ch03/Code/SqStack/Init.c}
\end{frame}

\begin{frame}[fragile]\ft{\tt PrintS.c}
  \lstinputlisting[
    language=C,
  ]{Chapters/Ch03/Code/SqStack/PrintS.c}
\end{frame}

\begin{frame}[fragile]\ft{\tt Push.c}
  \lstinputlisting[
    language=C,
  ]{Chapters/Ch03/Code/SqStack/Push.c}
\end{frame}

\begin{frame}[fragile]\ft{\tt Pop.c}
  \lstinputlisting[
    language=C,
  ]{Chapters/Ch03/Code/SqStack/Pop.c}
\end{frame}

\begin{frame}[fragile]\ft{\tt Clear.c}
  \lstinputlisting[
    language=C,
  ]{Chapters/Ch03/Code/SqStack/Clear.c}
\end{frame}

\begin{frame}[fragile]\ft{\tt Destroy.c}
  \lstinputlisting[
    language=C,
  ]{Chapters/Ch03/Code/SqStack/Destroy.c}
\end{frame}

